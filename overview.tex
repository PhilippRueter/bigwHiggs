% 
\IfFileExists{/u/r/dw580/Qspace/mypreamble}{ \input{/u/r/dw580/Qspace/mypreamble} }{ \input{/home/waalge/Qspace/mypreamble} } % Work and Laptop versions

\title{Overview talk}
\author{Marina Logares}
\date{}

\begin{document} 
\maketitle

Higgs bundles were introduced in 1987 Nigel Hitchen.
They are at the core of many mathematical and physical objects and theories.
In physics they appear when working with Yang-Mills equations, mirror symmetry, topological quantum field theory, supergravity.
In mathematics they appear in Teichmuller theory, integrable systems, and representations of $ \pi_1 (X) $. 

\begin{definition}
    A Higgs bundle $(E, \varphi) $ where $ E \rightarrow  X $ is a holomorphic bundle over a compact Riemann Surface. 
    $ \varphi \in H^0 ( X, \mathrm{End} (E \otimes K _X) ) $ where $K_X$ is the canonical bundle of $X$. 
\end{definition}

Let $X$ be an oriented Riemannian manifold, and $P \rightarrow X$ a principal $G$-bundle. 
Yang - Mills functional $YM$ is a functional on the space of connections of the associated adjoint bundle, defined by
\begin{equation}
    YM(A) = \int_X \mathrm{tr}(F_A \wedge * F_A) 
\end{equation}
where $ F_A= dA + [A, A] $ is the curvature of connection $A$, and $*$ is the Hodge star. 

In the case of four dimensions, if $F_A$ is dual or anti self dual then global minimums are attained.
Say that $A$ is an instanton if $ F_A = * F_A $, and has seen a lot of attention.
Solving $A$ in four dimensions is a struggle, so people have considered dimensional reductions. 
Reducing by supposing translational invariants in one dimension leads to magnetic monopoles studied by Atiyah and Hitchen.
A further one dimensional reduction, leads to Higgs bundles via the Higgs equations. 
\begin{align}
    \bar{\partial} \varphi & = 0 \\
    F_A + [ \varphi , \varphi^* ] &= 0 
\end{align}


Let $(E,\varphi) $ where $ E \rightarrow  X$ holomorphic, 
and $ E = H^\mathbb{C} $ bundle.
Let $ G$ be a non-maximal component. 
Then $ \varphi \in H ^0 ( X, E(\mathfrak{m}^ \mathbb{C} \otimes K) $
$\mathfrak{g} = \mathfrak{h} + \mathfrak{m} $ 

Then $ G = U(p,q) $ and $ E = V \oplus W $ , $\varphi = \begin{array}{rr}
    0 &  \gamma \\
    \beta & 0 \\
\end{array} $ 

Upcoming:
Non-abelian Hodge theory
Symplectic structure 
Hitchin map 
Singular Higgs bundles (parabolic bundles) 
Topology of moduli spaces 
Mirror symmetry for the character varieties. 

\section{Non abelian Hodge} % (fold)

Let $(E, \varphi) $ be our Higgs bundle over Riemann surface $X$.


$\varphi = 0 $. 
$M(r,d) $ is the moduli space of stable vector bundles of rank $r$ and degree $d$.
Narashimhain-Sashadin theorem. 
\begin{equation}
    M ( r, 0  ) \cong \{ \pi_ 1 (X) \rightarrow \mathrm{U}(n)\} / \mathrm{U}(n) 
\end{equation}

Donaldson generalises representation 
3 ingredients, $v$, $b$, connections $ \rho : \pi_1 ( X) \rightarrow  G$ 

\begin{equation}
     H^1 ( X, G) \cong H_1 ( X, G) \cong \mathrm{Hom} ( \pi / [\pi, \pi] , G) \cong \mathrm{Hom} ( \pi, G) 
\end{equation}

Non-abelian hodge theory consists of the following technique
\begin{equation}
    H^1 _{\Delta } (X, G) \cong H^1 _{dR} ( X,G) \cong H^1 _{\partial} ( X,G)  
\end{equation}
Relates the singular, d'Rham, and Dolbeaut cohomology. 
The topological, the differential, and the algebraic
% section (end)

\section{Symplectic Geometry} % (fold)

Suppose $ H^1( X, \mathbb{R} ) = \mathbb{R} $, 
for $ X $ compact Riemann surface of genus $g$. 
The cup product provides $H^1( X, \mathbb{R} ) $ with symplectic structure. 
Actually, 
$ \mathcal{M} ^s (r,d) $ the moduli of stable Higgs bundles is symplectic. 
\begin{equation}
    T_E M = H^1( X, \mathrm{End}(E) ) 
\end{equation}
by Serre duality
\begin{equation}
    T_E M ^* = H^0 ( X, \mathrm{E} \otimes K_X 
\end{equation}

$M^s$ is the moduli space of stable vector bundles 
Say that vector bundle $E \rightarrow  X$ is stable if
$\forall F \subset E $ 
\begin{equation}
    \mathrm{deg} (F) / \mathrm{rk} (F) < \mathrm{deg}(E) / \mathrm{rk}(E) 
\end{equation}
slope condition. 
In the case of Higgs bundles require stronger condition
\begin{equation}
     \forall F \subset E, ~~ \varphi(F) \subset F \otimes K 
\end{equation}
% section (end)


\section{Symplectic geometry} % (fold)

\begin{equation}
    \mathrm{dim} M = r^2 ( g-1 ) + 1
\end{equation}
% section (end)

\section{Hitchin map} % (fold)
($ G = \mathrm{GL} ( r, \mathbb{C} ) $) 
\begin{equation}
    \mathcal{M}^s \supset T^* M^s \rightarrow M ^s
\end{equation}

The hitchin map 
\begin{align}
    \mathcal{M}^s \rightarrow{h} \bigoplus _{i>0 } H^) ( X, K^i )  = H \\
    (E, \varphi ) \mapsto \mathrm{tr} ( \bigwedge ^i \varphi ) 
\end{align}
Proper map, with $ \mathrm{dim} H  = \frac{1}{2} \mathrm{dim} \mathcal{M} $
(cf spectral curves, mirror symmetry, and integral systems) 

SYZ- mirror symmetry 
% section (end)

\section{Parabolic} % (fold)

[NO IDEA]

% section (end)

\section{Topology of moduli of stable Higgs} % (fold)

Define a perfect morse function 
\begin{align}
    \mu : \mathcal{M} & \rightarrow  \mathbb{R} \\
    (E, \varphi)  \mapsto \| \varphi \| ^ 2 
\end{align}

\begin{equation}
    P _t ( \mathcal{M}) = \sum t ^{\mathrm{ind}} P _t ( \mathcal{N}) 
\end{equation}
where $\mathcal{N} $ are some critical somethings of something. 


% section (end)








\bibliographystyle{plain}
\bibliography{}


\end{document} 

