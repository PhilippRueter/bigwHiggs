
% 
\documentclass[10pt]{article}

\usepackage[margin=1in]{geometry}  % set the margins to 1in on all sides
\usepackage{graphicx}              % to include figures
\usepackage{amsmath}               % great math stuff
\usepackage{amsfonts}              % for blackboard bold, etc
\usepackage{amsthm}                % better theorem environments
\usepackage{ulem}                  % underline emphasize
\usepackage{tikz}                  % Graphics
\usepackage{tabularx}                  % tables
\usetikzlibrary{matrix}
\usepackage{enumitem} 
\setlist[enumerate]{topsep=0pt,itemsep=-1ex,partopsep=1ex,parsep=1ex}
\usepackage[all]{xy}

\theoremstyle{plain}
\newtheorem{theorem}{Theorem}[section]
\newtheorem{lemma}[theorem]{Lemma}
\newtheorem{proposition}[theorem]{Proposition}
\newtheorem{corollary}[theorem]{Corollary}
\newtheorem{conjecture}[theorem]{Conjecture}
\newtheorem{definition}[theorem]{Definition}
\newtheorem{example}[theorem]{Example}
\newtheorem{remark}[theorem]{Remark}
\newtheorem{question}{Question}[section]

\newtheorem*{theorem*}{Theorem}
\newtheorem*{lemma*}{Lemma}
\newtheorem*{proposition*}{Proposition}
\newtheorem*{corollary*}{Corollary}
\newtheorem*{conjecture*}{Conjecture}
\newtheorem*{definition*}{Definition}
\newtheorem*{example*}{Example}
\newtheorem*{remark*}{Remark}
\newtheorem*{question*}{Question}




\title{Character Varieties through examples}
\author{Andrea}
\date{}

\begin{document} 
\maketitle

Definition of character variety for any finitely generated group. 
Crash course in algebraic geometry.
State some useful results to do computations. 
Examples. 
Symplectic structure when $ \Gamma = \pi_1 ( \Sigma_g) $
Affine cubic curves. 

\section{Character varieties} % (fold)

\begin{definition}
    Let $ G \subset \mathrm{GL}_n (\mathbb{C} ) $ be complex reductive linear group, 
    and $ \Gamma $ a finitely generated group. 
    The character variety of $ \Gamma$ in relation to $G$ is 
    \begin{equation}
        \chi ( \Gamma, G) = \mathrm{Hom}( \Gamma, G) // G 
    \end{equation}
\end{definition}

% section (end)

\section{Algebraic geometry} % (fold)

\begin{definition}
    Affine variety  $ X \subset \mathbb{A}_\mathbb{C} ^n $.
\end{definition}

\begin{definition}
    Regular functions. 
\end{definition}

Examples: 
Suppose $X$ is the zero locus of the polynomials $p_1, \dots, p_k \in \mathbb{C} [x_1, \dots, x_n ] $. 
Then the regular functions are elements of ring 
\begin{equation}
    \mathbb{C}  [X] = \frac{\mathbb{C} [x_1, \dots, x_n ] }{(p_1, \dots, p_k)}
\end{equation}

The functor $\mathrm{Spec} $ takes quotient rings of the form above to their associated varieties. 

Let $ G $ act on the algebra $ A$. 
Denote the elements of $A$ fixed by $G$ by $A^G$. 
We define 
\begin{equation}
    X//G = \mathrm{Spec}( \mathbb{C} [X] ^G) 
\end{equation}
known as the categorical quotient. 

From now on $G = \mathrm{SL}_2 \mathbb{C} $. 

\begin{align}
    A(\Gamma) = \frac{\mathbb{C} [ M ^\gamma _{ij} ~|~ i,j =1,2~~ \gamma \in \Gamma]}{(\mathrm{det}M = 1, ~ M^{\gamma \delta}=M^{\gamma} M^{\delta})}
\end{align}

\begin{align}
    \mathrm{Hom}(\Gamma, \mathrm{SL}_2 \mathbb{C} ) = \mathrm{Spec}A(\Gamma) 
\end{align}

\begin{align}
    A(\Gamma) ^{\mathrm{SL}_2 \mathbb{C} } \rightarrow \chi(\Gamma) = \mathrm{Spec}(A(\Gamma) ^{\mathrm{SL}_2 \mathbb{C} }) 
\end{align}

Example: 
$\Gamma = \mathbb{Z} $. 
Then $\rho(gen) = \left[ \begin{array}{rr} a & b \\ c & d \\ \end{array} \right]  $ 
\begin{align}
    A(\mathbb{Z} ) = \frac{\mathbb{C} [a,b,c,d] }{ (ab-cd)=1) } 
\end{align}

\begin{align}
    A(\mathbb{Z} ) ^{ \mathrm{SL}_2 \mathbb{C} } = \mathbb{C} [a+d] \cong \mathbb{C} [z]
\end{align}
so character variety $ \chi(\mathbb{Z} ) = \mathbb{C}  $

Skein algebra 
\begin{align}
    B(\Gamma) = \frac{\mathbb{C} [ M^\gamma ~|~ \gamma \in \Gamma] }{(M^e = 2, M^\gamma M^\delta = M^{\gamma \delta} + M_{\gamma \delta ^{-1}} )}
\end{align}

[[ RED ?? DONT GET THIS ]] 
    \begin{equation}
        \chi_{\mathrm{Skein}} (\Gamma) = \mathrm{Spec} B(\Gamma) 
    \end{equation}

\begin{theorem}
    \begin{equation}
        \chi_{\mathrm{Skein}} ^{\mathrm{red}} (\Gamma) =  \chi ^{\mathrm{red}} ( \Gamma) 
    \end{equation}
\end{theorem}

\begin{corollary}
    If $A(\Gamma) $ and $B(\Gamma)$ are nice. 
    $\chi (\Gamma) \cong \chi_{\mathrm{Skein}}(\Gamma) $. 
\end{corollary}

\begin{theorem}
    $\Gamma = \pi_1 ( M) $ for $ M $ manifold (with boundary). 
    Then 
    \begin{equation}
        B(\Gamma)  = \bigoplus_ \gamma \mathbb{C} [\gamma] 
    \end{equation}
    where $\gamma$ runs over isotopy classes of multicurves (immersed 1-submanifolds not bounding a disc) 
\end{theorem}
[[Something funky here ]] 

Example:
$M = S^2 \setminus \{ p_1, p_2, p_3 \} $. 
Then $ \pi_1 = \mathbb{F} _2 $ the free group on two generators. 
Claim then that the multicurves are given by $a, b , a*b $ so 
\begin{equation}
    B(\Gamma) = \mathbb{C} [x_1, x_2, x_3] , ~~~ \chi( \mathbb{F} _2) = \mathbb{C} ^3 
\end{equation}

\begin{equation}
    \mathrm{Hom}(\mathbb{F} _2 , \mathrm{SL}_2 \mathbb{C} ) = ( \mathrm{SL}_2 \mathbb{C} ) 
\end{equation}

Have that $ \mathbb{C} [ \mathrm{SL}_2 \mathbb{C} ] ^{\mathrm{SL}_2 \mathbb{C} } $ 
is generated by some trace functions 
\begin{align}
    (A,B) \mapsto \begin{cases}
        \mathrm{Tr} ( A)  \\
        \mathrm{Tr} ( B) \\
        \mathrm{Tr} ( AB) 
    \end{cases}
\end{align}

% section (end)

\section{Affine cubic curves} % (fold)

Let $ M = S^2 \setminus \{p_1, \dots, p_4 \} $ 
then $ \pi_1 (M) = \mathbb{F} _3 $, $ \chi(\pi_1 ( M) ) = ( \mathrm{SL}_2 \mathbb{C} ) ^3 // \mathrm{SL}_2 \mathbb{C} $

$ A(\pi_1 ( M) ) ^{\mathrm{SL}_2 \mathbb{C} }  = \left<x,y,z, m_1, m_2, m_3, m_4 | \sim \right>$ 

where 
$ m_{1} = \mathrm{Tr}(M_{1} ) $, 
$ m_{2} = \mathrm{Tr}(M_{2} ) $, 
$ m_{3} = \mathrm{Tr}(M_{3} ) $, 
$ x = \mathrm{Tr}(M_{2}M_{3} ) $, 
$ y = \mathrm{Tr}(M_{1}M_{3} ) $, 
$ z = \mathrm{Tr}(M_{1}M_{2} ) $, 
$ m_{4} = \mathrm{Tr}(M_{1}M_{2}M_{3} ) $, 


It turns out that the
$ \mathrm{Spec}(A(\pi_1(M))^{\mathrm{SL}_2 \mathbb{C} } ) $ is a cubic hypersurface in $\mathbb{C} ^7 $
in the so called Frieke family 
Family of cubic surfaces 
\begin{align}
    xyz + x^2 + y^2 + z^2 + b_1 x + b^2 y + b_3 z +c = 0 
\end{align}

\begin{theorem}
    (Goldman - Toledo) 
    Every cubic surface in the Frieke family arises as a relative character variety.
\end{theorem}

% section (end)


\bibliographystyle{plain}
\bibliography{}


\end{document} 

