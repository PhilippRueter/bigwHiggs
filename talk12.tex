% 
\documentclass[10pt]{article}

\usepackage[margin=1in]{geometry}  % set the margins to 1in on all sides
\usepackage{graphicx}              % to include figures
\usepackage{amsmath}               % great math stuff
\usepackage{amsfonts}              % for blackboard bold, etc
\usepackage{amsthm}                % better theorem environments
\usepackage{ulem}                  % underline emphasize
\usepackage{tikz}                  % Graphics
\usepackage{tabularx}                  % tables
\usetikzlibrary{matrix}
\usepackage{enumitem} 
\setlist[enumerate]{topsep=0pt,itemsep=-1ex,partopsep=1ex,parsep=1ex}
\usepackage[all]{xy}

\theoremstyle{plain}
\newtheorem{theorem}{Theorem}[section]
\newtheorem{lemma}[theorem]{Lemma}
\newtheorem{proposition}[theorem]{Proposition}
\newtheorem{corollary}[theorem]{Corollary}
\newtheorem{conjecture}[theorem]{Conjecture}
\newtheorem{definition}[theorem]{Definition}
\newtheorem{example}[theorem]{Example}
\newtheorem{remark}[theorem]{Remark}
\newtheorem{question}{Question}[section]

\newtheorem*{theorem*}{Theorem}
\newtheorem*{lemma*}{Lemma}
\newtheorem*{proposition*}{Proposition}
\newtheorem*{corollary*}{Corollary}
\newtheorem*{conjecture*}{Conjecture}
\newtheorem*{definition*}{Definition}
\newtheorem*{example*}{Example}
\newtheorem*{remark*}{Remark}
\newtheorem*{question*}{Question}



\title{Spectral curves and integrability of Dolbeaut moduli}
\author{Philip}
\date{}

\begin{document} 
\maketitle

Rationale setup:
Moduli space $\mathcal{M}_{Dol} $ of stable Higgs Bundles.

[[ WHY IS THIS NOT Mhiggs?? ]]

Remark:
$\mathcal{M} \subset T^* \mathcal{N} $ open dense for $\mathcal{N} $ is the moduli space of stable vector bundles
Can generalise this: 
Let $ P \rightarrow  C $ be a principal $G$ - bundle, 
then let $V$ be an associated vector bundle to the fundamental representation.
Implicitly assuming that $G$ is $\mathrm{GL}_n \mathbb{C} $ ??

Claim: $ \mathcal{M} _{Dol} $ is an algebraically completely integrable system. 

Want: proof by construction 
Hitchin fibration $h : \mathcal{M} \rightarrow \mathcal{B}$ 
There exists curve $ \Sigma \rightarrow  C $ such that the fibres of the Hitchin fibration are $\mathrm{Jac}(\Sigma) $ spectral curves. 

\section{Integral Systems} % (fold)

Let $(M,\omega) $ be a (holomorphic) symplectic manifold. 

Example: Cotangent spaces, Kahler manifolds or hyperkahler manifolds. 
The latter is holomorphic symplectic $\Omega = \omega_J + i \Omega_K $. 

\begin{definition}
     For $ f \in C^{\infty} (M) $, the Hamiltonian of $f$ is the vector field $X_f $, such that 
     \begin{equation}
            df = \omega(X_f, \cdot) 
     \end{equation}
\end{definition}

\begin{definition}
    For $ f,g \in C^{\infty}(M)$, the Poisson Bracket $\{f, g\} = \omega(X_f, X_g) $. 
    Say $f,g$ Poisson commute if their Poisson bracket is 0. 
\end{definition}

Let f,g be $G$-invariant functions on $M$, then can define $\tilde{f}, \tilde{g}$ on quotient $\mu^{-1} (0)/ G $. 
They Poisson commute if and only if  $\tilde{f}, \tilde{g} $ Poisson commute. 

\begin{definition}
    Let $(M, \omega) $ of dim $2n$ is called (holomorphically) completely integrable system if:
    there exists $n$ functions $\{f_1, \dots, f_n \}$ which pairwise Poisson commute, 
    and are functionally independent ie $\Lambda_i df_i \neq 0 $ on an open dense set of $M$, denoted $M_0$. 
\end{definition}

Remark: 
The level sets of $\{f_i \} $ give a foliation on $M_0$ 

\begin{theorem}
    (Arnold - Louiville) 
    Let $(M, \omega, \{f_i \} ) $ be a (holomorphic) complete integrable system. 
    Let $N$ be a connected complement of the level set of $f$. 
    Then $ N$ is a diffeo (biholomorphic) to $ \mathbb{R} ^k \times T^{n-k} $ ( $ \mathbb{C} ^k \times T^{n-k} $) 

    In particular complete connected components are diffeomorphic (biholomorphic) to a torus. 
\end{theorem}

\begin{definition}
Algebraically complete integrable system is a holomorphic completely integrable system if the generic fibres of $f:M \rightarrow \mathbb{C} ^n $ are abelian varieties 
\end{definition}

% section (end)

\section{Hitchin Fibration} % (fold)

Consider $\mathcal{M}_{Dol} $. 

Want map $\varphi$ to a basis of $n = \mathrm{dim} (\mathcal{M}) /2 $ 
functions given by polynomials 
\begin{align}
    h: \mathcal{M} & \rightarrow  \mathcal{B} := \bigoplus _{i =1} ^r H^0 ( C, K ) 
    ( V, \varphi ) & \mapsto ( a_i (\varphi), \dots , a_r (\varphi) )
\end{align}

$\mathrm{det} ( \lambda - \varphi) = \lambda ^r  + a_1 \lambda^{r-1} +\dots + a_{r-1} \lambda + a_r $

Actually take $ \pi: K \rightarrow  C $ for $ \{ a_i \} \in \mathcal{B} $, 
let $ \varphi $ be the tautological section of $\pi^* K $. 
This map is proper and surjective and $ \mathrm{dim}(\bigoplus _i ^r H^{0} (C, K^i ) ) = n $. 
% section (end)

\section{Spectral Curve} % (fold)

Consider $ \mathrm{det}( \lambda - \varphi) = \mathrm{Char}_\varphi ( \lambda) $, 
gives an algebraic curve $\Sigma $ in $K = T^*C $ and is called a spectral curve. 
A generic curve (generic $\{a_i \} \in \mathcal{B}$ ) is smooth. 
We have map $ \pi: T^*C \rightarrow  C $, restricting to $\pi: \Sigma \rightarrow  C$. 

Thus we view $\Sigma$ as a ramified $r$ cover of $ C$. 
At generic points on $C$, the $\lambda_1, \dots, \lambda_r $ distinct . 
Eigenspaces give a line bundle $ L \rightarrow \Sigma $ implies a point in $\mathrm{Jac} (\Sigma) $

Claim: The fibre of $ h: \mathcal{M} \rightarrow \mathcal{B} $ are given by $\mathrm{Jac} (\Sigma) $ for $\Sigma = \Sigma_{\{a_i \}} $ 

Some technicalities about how all this work is omitted here. 
Use the direct image construction to get line bundle over $\Sigma$.

$u \subset C $ $H^0 (U, \pi_* L) = H^0 ( \pi^{-1} (U) , L) $

To retrieve $\varphi$ via the functorial structure.
[[ COMMUTATIVE DIAGRAM HERE ]] 

Stability: 
$\Sigma$ is irreducible. 
If there exists $M \subset V = \pi_* L $, 
$\varphi$ invariant then $ \mathrm{char}_\varphi |_M $ divides the character variety. 





% section (end)




\bibliographystyle{plain}
\bibliography{}


\end{document} 
