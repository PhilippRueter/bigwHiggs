% 
\documentclass[10pt]{article}

\usepackage[margin=1in]{geometry}  % set the margins to 1in on all sides
\usepackage{graphicx}              % to include figures
\usepackage{amsmath}               % great math stuff
\usepackage{amsfonts}              % for blackboard bold, etc
\usepackage{amsthm}                % better theorem environments
\usepackage{ulem}                  % underline emphasize
\usepackage{tikz}                  % Graphics
\usepackage{tabularx}                  % tables
\usetikzlibrary{matrix}
\usepackage{enumitem} 
\setlist[enumerate]{topsep=0pt,itemsep=-1ex,partopsep=1ex,parsep=1ex}
\usepackage[all]{xy}

\theoremstyle{plain}
\newtheorem{theorem}{Theorem}[section]
\newtheorem{lemma}[theorem]{Lemma}
\newtheorem{proposition}[theorem]{Proposition}
\newtheorem{corollary}[theorem]{Corollary}
\newtheorem{conjecture}[theorem]{Conjecture}
\newtheorem{definition}[theorem]{Definition}
\newtheorem{example}[theorem]{Example}
\newtheorem{remark}[theorem]{Remark}
\newtheorem{question}{Question}[section]

\newtheorem*{theorem*}{Theorem}
\newtheorem*{lemma*}{Lemma}
\newtheorem*{proposition*}{Proposition}
\newtheorem*{corollary*}{Corollary}
\newtheorem*{conjecture*}{Conjecture}
\newtheorem*{definition*}{Definition}
\newtheorem*{example*}{Example}
\newtheorem*{remark*}{Remark}
\newtheorem*{question*}{Question}



\title{Non Abelian hodge theory}
\author{Owen}
\date{}

\begin{document} 
\maketitle

Throughout $X$ is  a compact connected Riemann surface. 


\begin{theorem}
    (Non Abelain Hodge Theorem)
    \begin{equation}
        \mathcal{M}_{\mathrm{flat}}|^{\mathrm{ss}} \rightarrow  \mathcal{M}_{\mathrm{higgs}} 
    \end{equation}
\end{theorem}

where $Mflat$ is the moduli space of rank $n$ vector bundles over $X$ with flat connection. 
and $Mhiggs$ is the moduli space of rank $n$, degree $0$ vector bundles $E$ over $X$ together with a section $\varphi \in H^0 (X, \mathrm{End}(E) \otimes \Omega^1 _X) $. 

\begin{definition}
    $(E, \nabla)$ is simple if it has no proper flat subbundles. 
    It is semisimple if it decomposes as a direct sum of simple bunldes. 
\end{definition}
By $ Mflat|^{\mathrm{ss}} $ we denot the moduli space restricted to semisimple bundles. 

Assume $n=1$, so and $E$ is a line bundle. 

Recall that there is a comples analytic isomorphsm. 
\begin{align}
    Mbet & = \mathrm{Maps}(\pi_1 ( X) , \mathrm{GL}_n \mathbb{C} ) / \mathrm{GL}_n \mathbb{C}  \\
    & = \mathrm{Maps}(\pi_1 ( X) , \mathbb{C} ^*) \\
    & = \mathrm{Maps}(\pi_1 ( X)/[\pi_1(X) , \pi_1(X) , \mathbb{C} ^*) \\
    & = \mathrm{Maps}(H^1(X, \mathbb{Z} , \mathbb{C} ^*) \\
    & = H^1(X, \mathbb{C} ^*) 
\end{align}

In the case is a genus $g$ Riemann surface we have 
\begin{align}
    H^\bullet (X, \mathbb{Z} ) \begin{cases}
        things 
    \end{cases} 
\end{align}

Remark: when $n=1$ stability and semistability are automatic. 
Thus we have that NAHT is equivalent in our case to 
\begin{equation}
    H^1(X, \mathbb{C} ^*) \rightarrow Mhiggs
\end{equation}

Fact: there exists a complex manifold $\mathrm{Pic} ^0 $ called the Jacobian 
which parameterizes degree 0 line bundles.
In the case that $E$ is of rank 1 so the endomorphism bundle is trivial. 
Have higgs field $\varphi \in H^0 ( X , \Omega_X ^1 ) $ 

Thus we reformulate NAHT as 
\begin{equation}
    H^1(X, \mathbb{C} ^*) \rightarrow \mathrm{Pic}^0 (X) \times H^0 (\Omega_X ^1 ) 
\end{equation}

The usual hodge decomposition 
\begin{equation}
    H^1 ( X, \mathbb{C} ) = H^1 ( X, \mathcal{O} _X ) \oplus H^0 ( \Omega_X ^1 ) 
\end{equation}
which is almost what we want, but not quite. 

The rest of the talk aims to patch over some holes above, 
and complete the proof of NAHT.

\section{Sheaves} % (fold)

Consider the short exact sequence of sheaves of abelian groups 
\begin{align}
    0 & \mathbb{Z}  & \mathbb{C}  & \mathbb{C} ^* & 0 \\
    0 & \mathbb{Z}  & \mathcal{O}  & \mathcal{O}^* & 0 \\
\end{align}
And applying functor $H^ \bullet (X, - ) $ and consider the long exact sequence. 
We now play the diagram chasing game to prove the necessary injectivity and surjectivity results. 

Claim $ H^0 ( \mathbb{C} ^* ) \rightarrow  H^1 ( \mathbb{Z} ) $. 
Same holds for $ H^1 ( \mathbb{Z} ) \rightarrow  H^1( \mathcal{O} _X ) $.

By definition $ \mathrm{Pic} (X) = H^1 ( \mathcal{O} _X ) $
The $\mathrm{ker}( \mathrm{deg}: H^1 ( \mathcal{O}_X ^* ) \rightarrow H^2(\mathbb{Z} ) )$ is preciselt the Jacobian $\mathrm{Pic}^0 (X)$.

Surjectivity at $H^1( \mathbb{C} ^* ) $ follows from the universal soefficient theorem since 
\begin{equation}
    H^2 (\mathbb{Z} ) \rightarrow H^2 ( \mathbb{C} ) 
\end{equation}
is injective. 

Example: 
Suppose $X$ has genus 1. 
Then 
$H^1(\mathbb{Z} ) = \mathbb{Z} ^2 $, $ H^1 ( \mathbb{C} ) = \mathbb{C} ^2 $, $H^1 ( \mathcal{O} _X )  = \mathbb{C} $, 
$H^1(\mathbb{C} ^*) = (\mathbb{C} ^*) ^2 $, $H^0 (\Omega_X ^1) $. 

Warning: 
$H^1 ( \mathbb{C} ^* ) $ is not holomorphically equal to $\mathcal{M} _{\mathrm{flat}} $.

Suppose $H^1 ( \mathbb{C} ^* )$ splits as $ \mathrm{Pic} ^0 (X) \times H^0 (\Omega_X ^1 ) $. 
Then there is a section $s: \mathrm{Pic}^0 (X) \rightarrow  H^1 ( \mathbb{C} ^*) $. 
Pulls back to a map $ s: \mathrm{Pic}^0 (X) \rightarrow H^1 (\mathbb{C} ) $

[[ OTHER THINGS I CANNOT READ ]] 


% section (end)


Proof of $n=1 $ NAHT 

...









\bibliographystyle{plain}
\bibliography{}


\end{document} 

