% 
\documentclass[10pt]{article}

\usepackage[margin=1in]{geometry}  % set the margins to 1in on all sides
\usepackage{graphicx}              % to include figures
\usepackage{amsmath}               % great math stuff
\usepackage{amsfonts}              % for blackboard bold, etc
\usepackage{amsthm}                % better theorem environments
\usepackage{ulem}                  % underline emphasize
\usepackage{tikz}                  % Graphics
\usepackage{tabularx}                  % tables
\usetikzlibrary{matrix}
\usepackage{enumitem} 
\setlist[enumerate]{topsep=0pt,itemsep=-1ex,partopsep=1ex,parsep=1ex}
\usepackage[all]{xy}

\theoremstyle{plain}
\newtheorem{theorem}{Theorem}[section]
\newtheorem{lemma}[theorem]{Lemma}
\newtheorem{proposition}[theorem]{Proposition}
\newtheorem{corollary}[theorem]{Corollary}
\newtheorem{conjecture}[theorem]{Conjecture}
\newtheorem{definition}[theorem]{Definition}
\newtheorem{example}[theorem]{Example}
\newtheorem{remark}[theorem]{Remark}
\newtheorem{question}{Question}[section]

\newtheorem*{theorem*}{Theorem}
\newtheorem*{lemma*}{Lemma}
\newtheorem*{proposition*}{Proposition}
\newtheorem*{corollary*}{Corollary}
\newtheorem*{conjecture*}{Conjecture}
\newtheorem*{definition*}{Definition}
\newtheorem*{example*}{Example}
\newtheorem*{remark*}{Remark}
\newtheorem*{question*}{Question}




\title{Title}
\author{Ben}
\date{}

\begin{document} 
\maketitle

\section{BPS states} % (fold)

\begin{definition}
    BPS states or counts are invariants of 3CY categories. 
\end{definition}

A CY category is a category with a duality condition on associated algebras.

Examples of 3 CY categories:
\begin{enumerate}
    \item Coherent sheaves on $X$ where $X$ is a smooth projective 3-fold over $ \mathbb{C} $ 
        with trivial canonical bundle. 
        Serre duality $E, F \in \mathrm{Coh} (X) $ then $ \mathrm{Ext} ^i (E, F)  \cong \mathrm{Ext}  ( F, E) ^* $.
    \item Local systems on a real 3 dimensional closed orientable manifold. 
        IN this case $ H^i * (M, E^* \otimes F ) \cong H ^{3-i } ( M, F^* \otimes E^* ) ^* $ .
    \item $M$ as above. Then Representations of $ \mathbb{C} [ \pi_1 ( M) ] $. 
\end{enumerate}
Any one of these examples gives us the input of DT theory. 

Let $ C $ be 3CY category. 
Introduce the notion of charge lattice. 
Target for some [[SOMETHING ]]  invariant of algebras of $ \mathbb{C} $ 

Let $ N = \mathbb{Z}  $ - dimension of $ \mathbb{C} [ \pi ( M) ] $ - module 

$ \mathcal{M} _\gamma $ for $ \gamma \in N $ stack of objects  in $ C $ of class $ \gamma $. 
This is manageably small. 

Vanishing cycles: Say $\mathrm{Spec} ( \mathbb{C} [x]/ x^d ) $ is a moduli space of objects in some category $B$. 
This naively has a only a single object of a single point. 
However a deformation of $x^d $ will see the point split into $d$ objects. 
Thus we would like $ \# \mathrm{Spec} ( \mathbb{C} [x]/ x^d ) $ to be $ d$ and not $1$. 

Consider $ H^* ( X, \mathbb{Q} ) .... $ 
[[CANT SEE ]] 

DT theory is the study of the cohomology $ H( M ( C) , \mathcal{Q}  ) $ of the stack $M( C) $ over the category into sheaf $ \mathcal{Q} $  
We use the charge lattice and mixed hodge structures to decompose this space $D(MHS_N) $.

More precisely $[H(M(C), \mathcal{Q} ) ] \in K_0 (MHS_N) $ where 
$ K_0 ( MHS_N) = \mathbb{Z} [[L]] / ([L'] +[L''] = [L] ) $ if $ L' \rightarrow L \rightarrow L'' \rightarrow \Delta \in D(MHS_N) $. 
This is huge and unwieldly. 

How to even write this down? 
Answer: via Plethystic exponential $\mathrm{EXP} $, 
$f = \sum_{d \in \mathbb{Z} ^m } a_d x^d $ , 
$ \mathrm{EXP} (f) = \prod_{d \in \mathbb{Z} ^m } ( 1 - x^d ) ^{-a_d } $ . 

Where does this come from?
Origin: consider 
\begin{align}
    K ( \mathrm{Vect} _{\mathbb{Z} ^m } ) \xrightarrow{\chi} \mathbb{Z} [[\mathbb{Z} ^m ]] \\
    [v] \mapsto \sum _d \mathrm{dim} V_d x^d 
\end{align}
but this ill defined... 

cf Grothendeick something. 
Commutative diagramme between characters and symmetric algebra maps. 

[[ ERRRRRRR ]]

Question: How does on understand $[ H(M(C), \mathcal{Q} ] $?
Its just too big. 

Integrality conjecture (historically misnamed):
This says that $ [H ( M (C), \mathcal{Q} ) ]  = \mathrm{EXP} ( \oplus_{\gamma} BPS_\gamma \otimes H_{\mathbb{C} ^* }  () $ 
where $ BPS _\gamma $ is a finite dimensional mixed hodge structures. 

Many stupid things have appeared in this talk. 
We should try to address these...

Let $ M $ be an $n$ dimensional orientable closed real manifold. 
$ \mathbb{C} [\pi_1 ( M) ]$ -modules. 
$n$CY categories, with $ \mathrm{Ext}^i (E,F) \cong \mathrm{Ext}^{n-i} ( F, E ) ^* $
we want $ n = 2 $ not $3$. 

\begin{theorem}
    (BEN) 
    Let $B$ be a 2CY category. 
    Let $ C $ be a category of pairs $(E,f)$ for $ E \in B $, and $ f: E \rightarrow E^* $.
    Then $C$ is 3CY  and $H ( M(C) , \mathcal{Q} ) \cong H( M( B) , \mathbb{Q} ) $ 
\end{theorem}

Conclusion: $[ M(\mathbb{C} [\pi_1 ( \Sigma_g) ] ) ] = \mathrm{EXP} ([\oplus _n BPS_{n,g} \otimes H_{\mathbb{C} ^* } ( pts) ] ) $

Hausel Villegas : $[BPS _{n,g} ] = [ H( \mathrm{Rep}_n ^1 ( \Sigma_g) ] $ 

$ \mathrm{Rep}_n ^1 (\Sigma_g ) = [A, B ...] / \mathrm{PGL}_n $

and by NAHT $ \cong \mathrm{Higgs}_{n,1} ( C) $ 

Categorified Integrability theorem. 
this is much much stronger. 

What does this have to do with the $P=W $?
Only conjectures to follow?

Conjecture: 
$ BPS_{g,n} \cong H( \mathrm{Rep} _n ^1 ( \Sigma_g ) $

Theres analogue of the conjecture on the Higgs side. 
There's a network of conjectures. 

Conjecture: 
$H(\mathrm{Rep} ( \mathbb{C} [\pi_1(\Sigma_g)] )) \cong \mathrm{Sym} ( \bigoplus H(\mathrm{Rep}_n ^1 ( \Sigma_g ) \otimes H_\mathbb{C} (pt) ) ) $

$ H(\mathrm{Higgs} _{0} ^{ss} ) \cong \mathrm{Sym} ( \bigoplus H( \mathrm{Higgs} ( C)  \oplus H_{\mathbb{C} ^*} (pt) )) $ 

If we believe these conjectures, 
then $P = W $ becomes equivalent to the an isomorphism between the two LHS above ( call it $\alpha$). 
$\alpha$ is an isomorphism between 2 qunantum groups. 
They contain lie algebra structures on respective BPSg and BPS-Higgs 

Claim: $P, W$ filtrations come from induced $\mathfrak{sl}_2 $ on these Lie algebra. 


% section (end)





\bibliographystyle{plain}
\bibliography{}


\end{document} 

