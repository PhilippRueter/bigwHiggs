% 
\documentclass[10pt]{article}

\usepackage[margin=1in]{geometry}  % set the margins to 1in on all sides
\usepackage{graphicx}              % to include figures
\usepackage{amsmath}               % great math stuff
\usepackage{amsfonts}              % for blackboard bold, etc
\usepackage{amsthm}                % better theorem environments
\usepackage{ulem}                  % underline emphasize
\usepackage{tikz}                  % Graphics
\usepackage{tabularx}                  % tables
\usetikzlibrary{matrix}
\usepackage{enumitem} 
\setlist[enumerate]{topsep=0pt,itemsep=-1ex,partopsep=1ex,parsep=1ex}
\usepackage[all]{xy}

\theoremstyle{plain}
\newtheorem{theorem}{Theorem}[section]
\newtheorem{lemma}[theorem]{Lemma}
\newtheorem{proposition}[theorem]{Proposition}
\newtheorem{corollary}[theorem]{Corollary}
\newtheorem{conjecture}[theorem]{Conjecture}
\newtheorem{definition}[theorem]{Definition}
\newtheorem{example}[theorem]{Example}
\newtheorem{remark}[theorem]{Remark}
\newtheorem{question}{Question}[section]

\newtheorem*{theorem*}{Theorem}
\newtheorem*{lemma*}{Lemma}
\newtheorem*{proposition*}{Proposition}
\newtheorem*{corollary*}{Corollary}
\newtheorem*{conjecture*}{Conjecture}
\newtheorem*{definition*}{Definition}
\newtheorem*{example*}{Example}
\newtheorem*{remark*}{Remark}
\newtheorem*{question*}{Question}



\title{Hodge theory}
\author{Marille Ong}
\date{}

\begin{document} 
\maketitle

\section{Four worlds of Riemann surfaces} % (fold)

\begin{definition}
  A Riemann surface $X$ is a complex manifold of dimension $1$.
\end{definition}

examples:
$ \mathbb{P} $.... 

Denote two sheaves of particular interest. 
For open subset $U \subset X $, 
\begin{align}
    \mathcal{O}_X ( U) =  \mbox{ holomorpic functions on $U$ } \\
    \mathcal{M}_X ( U) =  \mbox{ meromorphic functions on $U$ } 
\end{align}

To be Tabulated 

\begin{tabular*}{\textwidth}{c|cccc}
   &   Topological  & Differential & Dolbeaut & algebraic \\
    Classification results & 
    Every closed orientable surface is homeomorphic to a sphere with $g$ handles.  & 
Let $M $ be a smooth orientable closed manifold dimension $2$, 
    then $M$ is diffeomorphic to $\Sigma _g $. & 


Denote this by $\Sigma_g$. 
Pictorially...

Have descriptions of the fundamental group
    $\pi_1 ( \Sigma_g) = \left< a_i, b_i | i =1, \dots, g ~~ a_i b_i a_i^{-1} b^{-1} \right> $

Differential 

D'rham theorem 


Dolbeaut + Algebraic.
Holomorphic objects relate to algebraic varieties, 
Holomorphic maps relate to regular morphisms, 
Riemann surfaces relate to projective curves. 
\end{tabular*} 

Example: Complex tori relate elliptic curves. 
$ \Gamma( 1, \omega) $  and $j$-invariants 

% section (end)

\section{Line Bundles} % (fold)

\begin{definition}
A holomorphic line bundle $ L \rightarrow  X$ is a complex vector bundle. 
    (Axioms stated here)
\end{definition}

The set of (holomorphic) line bundles (upto isomorphism) forms a group with tensor power acting as the binary operator.
The trivial bundle is then the identity element, and inverse is given by duality.
Denote this group $ \mathrm{Pic}(X) $. 

\begin{definition}
    Let $L$ be a line bundle with trivialisation $\{ U_i \} $ 
    A holomorphic section $s$ of line bundle ... (Axioms stated)
    A meromorphic section $s$ of line bundle ... (Axioms stated)
    $\mathcal{O}_L $ denotes the sheaf of holomorphic sections. 
\end{definition}

\begin{definition}
    A sheaf $ \mathcal{F} $ on $ X$ is an invertible if for all $U_i$  in the cover $\{U_i\} $ 
    such that $ \mathcal{F}_i$ is a free $\mathcal{O}_{U_i} $ module. 
\end{definition}

\begin{theorem}
    There exists a canonical isomorphism 
    \begin{equation}
        \mathrm{Pic} (X)  \cong H^1 ( X, \mathcal{O}^* _ X ) 
    \end{equation}
    And isomorphic to the group of invertible shead on $X$ under $ \otimes$. 
\end{theorem}

\begin{definition}
    The canonical bundle $ K_X $ is the determinant of the cotangent bundle. 
\end{definition}

\begin{theorem}
    (Riemann - Roch) 
    \begin{equation}
        h^0 (X, \mathcal{O} _L) - h^0 (X, \mathcal{O}_{L^* \otimes K_X}) = \mathrm{deg} (L) + 1 - g 
    \end{equation}
\end{theorem}

\begin{theorem}
    (Serre duality) 
    \begin{equation}
        H^k ( X, \mathcal{O}_L ) \cong H^{n-k} (X, \mathcal{O}_{L ^* \otimes K}) ^* 
    \end{equation}
\end{theorem}


% section (end)

\section{Divisors} % (fold)

\begin{definition}
    A divisor $D : X \rightarrow \mathbb{Z} $ which is 0 for all but finitely many $x \in X$. 
    Represent $D$ as the formal sum $ \sum D(x) x $. 
    The set of all divisors is denoted $ \mathrm{Div}(X)$
    The degree of a divisor $D$ is $\mathrm{deg}(D)  = \sum D(x) $. 
    For any function $f$ have the associated divisor $ \mathrm{Div}(f) = \sum \mathrm{ord}_x(f) x $  
\end{definition}

Divisors form a group under formal addition in a natural way. 
We induce on the set of divisors a partial ordering induced by $\mathbb{Z} $. 
There are notable subgroups.

Say two divisors are equivalent if they differ by a principle divisors 
The class group is defined as 
\begin{equation}
    \mathrm{Cl} ( X) = \frac{\mathrm{Div }(X)}{\mathrm{PDiv}(X) }  
\end{equation}
And the analogue for degree 0 .

For a line bundle we can define the associated divisor, which is defined upto principle divisor. 
That is ...
\begin{theorem}
 There exists 
    \begin{equation}
        \frac{\mathrm{Div}(X) }{\mathrm{PDiv}(X) }  \cong \mathrm{Cl}(X) \cong \mathrm{Pic}(X) 
    \end{equation}
    And the $0 $ analogue. 
\end{theorem}

Weiestrass problem. 
Given divisor $ D$ ....
% section (end)

\section{Jacobian} % (fold)

Define $H^0 ( X, \Omega_X ^ 1) $ space of holom !-forms.
Define 
\begin{align}
    \lambda _ C : H^0 ( X, \Omega_X ^1 ) & \rightarrow  \mathbb{C} \\
    \omega & \mapsto \int _C \omega
\end{align}
By Stokes $\lambda$ only depends on the class $[C] \in H_1 (X, \mathbb{Z} ) $ of curve $C$. 

Lettig $ \Lambda$ be the image of $\lambda$.

\begin{definition}
The jacobian of $ X$ is given by 
    \begin{equation}
        \mathrm{Jac}(X) = H^0 (\Omega^1 _X )/ \Lambda
    \end{equation}
    Thus we identify $H^0 ( X, \Omega^1 _X) ^* $ with $ \mathbb{C} ^ g$
\end{definition}

\begin{definition}
    Fix a base point $ p_0 \in X $ define 
    \begin{align}
        A: X & \rightarrow \mathrm{Jac}(X) \\
        p & \mapsto \int \gamma_p \omega 
    \end{align}
    where $ \gamma_p $ is a path $p_0 $ to $p $. 
    Abel-Jacobi map.

    Extend linearly to $ A : \mathrm{Div}(X) \rightarrow \mathrm{Jac} (X) $.
\end{definition}

\begin{theorem}
    (Abel) If $D \in \mathrm{Div}_0 $ then $A_0 (D) = 0 $ iff $D \in \mathrm{PDiv}(X)$
\end{theorem}

\begin{theorem}
    (Jacobi) The map $A_0 $ is surjective and $\mathrm{Pic} _0 \cong \mathrm{Jac}(X)$ 
\end{theorem}

% section (end)

\section{Hodge theory } % (fold)

Let $X$ now be compact complex manifold. 
The induced almost complex structure (here defined) 
We have the decomposition of the exterior algebra into $(p,q)$ forms. 
Denote the sheaf section of $ \Lambda^{p,q} X$ by $ A^{p,q} (X) $ 

We can then decompose $ d$ into $ \partial$ and $ \bar{\partial} $ 

\begin{theorem}
    (Dolbeault) 
    \begin{equation}
        H^{p,q} ( X) \cong H^q ( X, \Omega ^p _X ) 
    \end{equation}
\end{theorem}

Really long list defining the operators 

\begin{theorem}
    (Hodge theorem) 
\end{theorem}

\begin{theorem}
    $ \mathcal{H} ^{p,q} _{\bar{\partial} } (X) \cong H^{p,q} (X) $ 
\end{theorem}

\begin{theorem}
    $H^k _{\Delta } ( X, \mathbb{C} ) = \bigoplus _{p+q = k }  H^{p,q} ( X) $
    In the case where $ k = 1 $
    $H^1 _\Delta ( X, \mathbb{C}  ^* ) \cong \mathrm{Jac} (X) \oplus H^0 ( X, \Omega_X ^ 1) $
\end{theorem}


Hodge theory gives us the set of 1 dim representations $\pi_1(X) $ is isomorphic to the set of holomorphic line bundles with degree 0 

% section (end)


\bibliographystyle{plain}
\bibliography{}


\end{document} 

