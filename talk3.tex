% 
\documentclass[10pt]{article}

\usepackage[margin=1in]{geometry}  % set the margins to 1in on all sides
\usepackage{graphicx}              % to include figures
\usepackage{amsmath}               % great math stuff
\usepackage{amsfonts}              % for blackboard bold, etc
\usepackage{amsthm}                % better theorem environments
\usepackage{ulem}                  % underline emphasize
\usepackage{tikz}                  % Graphics
\usepackage{tabularx}                  % tables
\usetikzlibrary{matrix}
\usepackage{enumitem} 
\setlist[enumerate]{topsep=0pt,itemsep=-1ex,partopsep=1ex,parsep=1ex}
\usepackage[all]{xy}

\theoremstyle{plain}
\newtheorem{theorem}{Theorem}[section]
\newtheorem{lemma}[theorem]{Lemma}
\newtheorem{proposition}[theorem]{Proposition}
\newtheorem{corollary}[theorem]{Corollary}
\newtheorem{conjecture}[theorem]{Conjecture}
\newtheorem{definition}[theorem]{Definition}
\newtheorem{example}[theorem]{Example}
\newtheorem{remark}[theorem]{Remark}
\newtheorem{question}{Question}[section]

\newtheorem*{theorem*}{Theorem}
\newtheorem*{lemma*}{Lemma}
\newtheorem*{proposition*}{Proposition}
\newtheorem*{corollary*}{Corollary}
\newtheorem*{conjecture*}{Conjecture}
\newtheorem*{definition*}{Definition}
\newtheorem*{example*}{Example}
\newtheorem*{remark*}{Remark}
\newtheorem*{question*}{Question}



\title{Vector bundle over Riemann surface}
\author{Charles}
\date{}

\begin{document} 
\maketitle

\begin{definition}
    Let $M$ be a Riemannian manifold 
    then a smooth vector bundle $(E, M, V, \pi) $ such that the projection 
    $ \pi : E \rightarrow  M $ is smooth and 
    there exists a cover of $M$ such that all $ p \in M $, $ \pi^{-1} (U_i) \cong U_i \times V $, for vector space $V$. 
\end{definition}

We get transition maps $ \{ \varphi_{ij} : U_{ij} \rightarrow  \mathrm{GL} ( V) $ satisfying the cocycle condition.

[[ STATE CONDITIONS ]]

From a cover $\{U_i \} $ and transition maps $ \varphi _{ij} $ we can recover the vector bundle. 

[[ EXPLICITLY HOW? ]]

Examples: 
\begin{enumerate}
    \item Trivial. 
    \item Cotangent space. 
\end{enumerate}

In the case $ E$ is of rank 1, then call $E$ a line bundle. 
For $ \xi \in H^1( X, \mathcal{O} ^* ) $ holomorphic line bundles. 

Vector bundles from fundamental group representation of $X$, for $X$ a Riemann surface. 

[[ WHAT IS THIS SUPPOSED TO MEAN HERE ]] 

$\rho : \pi_1 ( X) \rightarrow  \mathrm{GL} (V) $, 
$ \hat{X} \times _\rho V : = \hat{X} \times V / \sim $
where $ \hat{X} $ is the universal cover of $ X $ .
And the equivalence is as it should ....


\begin{definition}
    As smooth section $ s: X \rightarrow  E $ is a smooth map such that $ \pi \circ s  = \mathrm{id} | _ X $ 
\end{definition}

$\Gamma ( TX) $ denotes the space of vector fields.

We can construct new bundles from existing of ones with $ \oplus, \otimes$ and $ \wedge $. 
Direct some, tensor product, exterior products 

\begin{definition}
    Let $E \rightarrow X $ and $ E' \rightarrow  X' $ be smooth vector bundles over $ X$. 
    Then a bundle map is a pair $(f,u) : (E,X) \rightarrow  ( E', X') $ such that the relevant maps commute. 
    The set of all such is $\mathrm{Hom}(E, E') $ and pro... () 
\end{definition}

If we have a map $ f: Y \rightarrow X $ then we can consider the pullback of $ e \rightarrow X $. 


\begin{proposition}
    Let $E \rightarrow X $ be a smooth bundle, let $ f: Y \rightarrow  X $ 
    and $ g: Y \rightarrow X $ be homotopic then $ f^* E \cong g^* E $, 
    topological classification of rank $n$ vector bundles over $ X $ 
\end{proposition}

\begin{definition}
    The Grassmanian $G(n, n+k) $ is the space $n$-planes in $ \mathbb{R} ^{n+k} $
\end{definition}

\begin{definition}
    The Tautological bundle $ \gamma^n ( \mathbb{R} ^{n+k} ) $ over $ G(n, n+k) $
     to be the bundle with elements are of the form $ (V, v) $ where $ v \in V$. 
\end{definition}

\begin{theorem}
    Any rank $n$ vector bundle over a paracompact space admits a bundle map to $ \gamma^n $. 
    Any two bundle maps are isomorphic. 
\end{theorem}

Have a correspondence smooth rank $n$ vector bundles over $ X $ 
and the homotopy classes of smooth maps from $ X $ to $ G(n,n+2) $. 

Degree for line bundles look at the degree of the associated divisor. 
For higher rank, do as above for $ \mathrm{det} ( E) $ . 


Over $ X$ a Riemann surface. 
The trivial complex line bundle with $ 2$ open sets. 
Pick  $p_0 \in X $, $U_1 $ is a contractible neighbourhood of $p$.
$ U_2  = U_1 \setminus \{p\} $. 

[[ WHERE IS THIS USED ]] 

$f : S^1 \rightarrow S^1 $ 

Have short exact sequence 
\begin{equation}
    0 \rightarrow  \mathbb{Z}  \rightarrow  C^\infty \xrightarrow{\mathrm{exp}} C^{\infty *}
\end{equation}
Get the long exact sequence which results in 
\begin{equation}
    H^1 ( X, C^\infty ) \rightarrow H^2( X, \mathbb{Z} ) \cong \mathbb{Z} 
\end{equation}
which is otherwise known as the degree map

\begin{definition}
    A connection $ D: \Omega^0 ( X, E) \rightarrow  \Omega ^1 ( X, E) $ is a linear map satisfying a Leibniz rule:
    for $ \sigma \in \Omega^0 ( X, E) $ and $f \in C^\infty $, then 
    $ D(f \sigma) = df \otimes \sigma + f D\sigma$. 
\end{definition}

Pick a local frame. 
Then there exists some $\omega$ such that $ D = d + \omega $. 
$\omega $ is known as the connection 1-form. 

$\omega $ satisfies the computational condition that changing frames by $g$ 
\begin{equation}
    \omega _ {e'} = dg \cdot g^{-1} + g \omega_e g^{-1} 
\end{equation}

\begin{definition}
    A hermitian structure $H$ on a complex vector bundle $ E$ is a smooth family of 
    Hermitiain scalar products on $E$ such that for any sections $ s, s'$, $ H(s,s') $ is smooth. 
\end{definition}

\begin{definition}
    A hermitian structure $H$ is a unitary with respect to a connection $D$ if 
    for $ s_1, s_2 \in \Omega^0 (E) $ we have $ dH(s_1, s_2 ) = H(Ds_1, s_2) + H(s_1, Ds_2)$.
\end{definition}

\begin{definition}
    $E \rightarrow  X$ is a holomorphic vector bundle if $ \pi: E \rightarrow X $ is holomorphic. 

\end{definition}

\begin{proposition}
    The following are equivalent. 
    1. $ \pi : E \rightarrow X $ is holomorphic.
    2. $ \varphi_{ij} : U_{ij} \rightarrow \mathrm{GL}(U) $ holomorphic
    3. $ \exists \bar{\partial} _E : \Omega ^0 ( E) \rightarrow \Omega ^{0,1} ( E) $ 
    satisfying the Leibniz rule, and $ \bar{\partial} ^2  = 0 $ .
\end{proposition}

\begin{theorem}
    Let $ (E, \bar{\partial}_E ) $ be a holomorphic vector bundle. 
    Let $H$ be a holomorphic structure on $E$ (??). 
    Then there exists a connection that is unitary with respect to $H$, 
    and $D^{0,1} = \bar{\partial}_E$.
\end{theorem}

Where $ D^{0,1} = \pi^{0,1} \circ D$.

[[ EXPAND ON PROJECTIONS DEFINITION ]] 

Have $ J : TX \rightarrow TX $. 
Proof: 
Calculation using connection 1-form $ \omega \partial H \cdot H^{-1}$
and parallel transport. 

\begin{definition}
    A section is parallel with respect to a connection $ D$ if $Ds = 0 $
\end{definition}

Parallel transport is, in general, sensitive to the underlying curve. 
It depends on the underlying curvature of the connection. 

\begin{definition}
    The curvature of $D$ is given by $D^2$ 
\end{definition}
This is sensible by extending the initial definition of $ D$ to the exterior algebra. 
Say $D$ is flat, if $ D^2 = 0 $. 

\begin{theorem}
    If $D$ is flat then the holonomy only depends on the homotopy of the curve. 
\end{theorem}

[[ CONCLUDING REMARK ?? ]] 

\bibliographystyle{plain}
\bibliography{}


\end{document} 

