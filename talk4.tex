% 
\documentclass[10pt]{article}

\usepackage[margin=1in]{geometry}  % set the margins to 1in on all sides
\usepackage{graphicx}              % to include figures
\usepackage{amsmath}               % great math stuff
\usepackage{amsfonts}              % for blackboard bold, etc
\usepackage{amsthm}                % better theorem environments
\usepackage{ulem}                  % underline emphasize
\usepackage{tikz}                  % Graphics
\usepackage{tabularx}                  % tables
\usetikzlibrary{matrix}
\usepackage{enumitem} 
\setlist[enumerate]{topsep=0pt,itemsep=-1ex,partopsep=1ex,parsep=1ex}
\usepackage[all]{xy}

\theoremstyle{plain}
\newtheorem{theorem}{Theorem}[section]
\newtheorem{lemma}[theorem]{Lemma}
\newtheorem{proposition}[theorem]{Proposition}
\newtheorem{corollary}[theorem]{Corollary}
\newtheorem{conjecture}[theorem]{Conjecture}
\newtheorem{definition}[theorem]{Definition}
\newtheorem{example}[theorem]{Example}
\newtheorem{remark}[theorem]{Remark}
\newtheorem{question}{Question}[section]

\newtheorem*{theorem*}{Theorem}
\newtheorem*{lemma*}{Lemma}
\newtheorem*{proposition*}{Proposition}
\newtheorem*{corollary*}{Corollary}
\newtheorem*{conjecture*}{Conjecture}
\newtheorem*{definition*}{Definition}
\newtheorem*{example*}{Example}
\newtheorem*{remark*}{Remark}
\newtheorem*{question*}{Question}



\title{GIT Versus symplectic reduction}
\author{Soheyla Feyzbakhsh}
\date{}

\begin{document} 
\maketitle
\section{The problem } % (fold)

Suppose we have group $G$ acting on space $X$. 
We would like to describe $ X/G $ of $ G$-orbits and inherit properties of $X$, 
however there exists bad points. 

Example: 
Consider $ G = \mathbb{C} ^* $, $ X = \mathbb{C} ^2 $ with action 
\begin{equation}
    s(x,y) \mapsto (sx, sy) 
\end{equation}
The quotient $ \mathbb{C} ^2 / \mathbb{C} ^* $ is non-hausdorff as some orbits are not closed. 

Alternatively suppose
\begin{equation}
    s(x,y) \mapsto (sx, s^{-1} y) 
\end{equation}
Now both axis are in the limit of $xy- \alpha $ as $\alpha \rightarrow 0 $.

Both GIT and symplectic reduction choose some `unstable' orbits and deal with them. 
% section (end)

\section{GIT} % (fold)


Let $ G$ act on space $X$ where $ \mathrm{SL} ( n , \mathbb{C} ) $ and $ X \subset \mathbb{P} ^n $.
The topological characterisation of semi-stability. 
Let $X \subset \mathbb{P} ^n $ has associated affine $ \tilde{X} \subset \mathbb{C} ^{n+1} $. 
$G$ acts in $ \tilde{X} $. 
so for any $ x \in X $, pick $ \tilde{x} \in \tilde{X} $ in lift.
Say that $x$ is stable if $0 \notin \overline{G . \tilde{x} } $
Say that $x$ is polystable if $G . \tilde{x}  $ is closed 
Say that $x$ is stable if it is polystable and has a finite stabliser.

\begin{theorem}
    There exists a projective variety $( X//G ) $ such that there exists a surjective morphism 
    $\varphi : X^{ss} \rightarrow X//G $ which is a good git. 
    That is 

    [[ DEFINITION OF GIT ]] 
\end{theorem}

Consider our earlier example $ \mathbb{C}  ^* $ acts on $ \mathbb{C} ^2 $ now sat in $ \mathbb{P} ^2$
then 
$s(x:y:z) \mapsto ( sx: sy: z) $. 
We have three types of orbit, 
$(x,y,z) \mapsto (x,y,z/s) $ then $ 0 \in \overline{G . ( 0,0,1) } $ so $ X// G = (\mathbb{C} ^2 \setminus \{0\} / \mathbb{C} ^* \cong \mathbb{P} ^1 $
$(x,y,z) \mapsto (sx,sy,z) $ then $ X// G $ is just one point.
$(x,y,z) \mapsto (s^2x,s^2 y, sz) $ then $ X// G $ is empty.

Here there is choice of embedding, which results in different $X//G$. 

An alternative definition runs along the following lines. 

Let $X \subset \mathbb{P} ^n $ and $ G \subset \mathrm{PSL} (n) $ such that $ G $ acts on $ X $.
Let $ K = \Gamma (\bigoplus_{k \in \mathbb{N} }  \mathcal{O}(k)|_X  ) $. 
Set $ K ^G $ be the set of elements invariant under $G$. 
Then the git reduction of $ X$ by $G$ is simply $ \mathrm{Proj}(K^G)$ 

[[ I DO NOT UNDERSTAND  QUITE WHY WE DID ALL THIS ]] 

[[ CF WITH PURELY ALGEBRAIC ROOT?? ]] 

% section (end)

\section{Symplectic reduction} % (fold)


Let $ K = G \cap \mathrm{SU}(n+1) $ 
The action of $ K $ on $ X$ is smooth, but also symplectic and Hamiltonian. 

[[ RECALL WHAT THIS MEANS ]]

Have 
\begin{align}
    \mathfrak{k} & \rightarrow  C^\infty ( \mathbb{P} ^n , \mathbb{R} ) \\
    v & \mapsto [ m_v] 
\end{align}

Put together all of the Hamiltonians $ m_v $ to give a moment map $ m : X \rightarrow  \mathrm{Lie}(K) ^* $
such that $\left< m(x) , v \right> = m_v(x) $, for all $ v \in \mathrm{Lie} (K) $. 

A moment map is unique up to an addition of a central element in $\mathrm{Lie} (L) ^*$ 

\begin{theorem}
    (Marsden - Weinstein - Meyer) 
    If the action of $K$ on $m^{-1}(0) $ is free and proper, 
    then the symplectic reduction $ X ^{red} = m^{-0} /K $ 
    is a symplectic manifold with dimension $ \mathrm{dim}(X) - 2 * \mathrm{dim}(K) $. 
\end{theorem}


Consider one of the examples above. 
$K = \mathrm{U} (1) $ ...
% section (end)

\section{Relating reductions} % (fold)


\begin{theorem}
 (Kempf- Ness) 
    A $ G $ - orbit contains a zero of the moment map if and only if it's polystable. 

    If $ x \in X $ is polystable, then if the orbit $ G . x $ meets $ m^{-1} (e) $ in a single $K$-orbit.

    $x \in X $ is semistable if and if its orbit closure meets $ m^{-1} (0) $. 

    $m^{-1} ( 0) \subset X ^{ss} $ which gives a homeomorphism 
    $m^{-1} (0) / K \rightarrow X//G $
\end{theorem}


Moduli of Vector bundles over $ (X, \mathcal{L} ) $
compact ?? moduli space we have to consider coherent sheaves $E$ of the same hilbert polynomial 

[[ WHAT IS HILBERT POLYNOMIAL ]] 

For any coherent sheaf $E$  $E(r) \otimes L^{otimes r} $, 
then for $ r >> 0 $, $E(r) $ is generated by global sections and has no higher cohomology. 
\begin{equation}
    0 \rightarrow \mathrm{\varphi} \rightarrow \mathcal{O} _X ^{\bigoplus h^0} (E(r)) \xrightarrow{\varphi} E(r) \rightarrow 0 
\end{equation}

\begin{equation}
    \chi(E) = \mathrm{dim} H^0 (X, E(r)) 
\end{equation}

We fix an isomorphism $H^0 (E(r) ) \cong \mathbb{C} ^N $ where $ N = \chi(E(r))$
then all $ E's $ are a quotient of $ \mathcal{O} (-r) {\oplus N } $ which are parameterised by a subset of the Grassmanian. 

Subset $ H ^0 ( \mathrm{Ker}(\varphi(s) ) \subset H^0 ( \mathcal{O} (s) ^{\otimes N} )$ 
So we divide by the choice of isomorphism. 
That is $ \mathrm{SL}(N, \mathbb{C}  ^N ) $ to get a moduli space of semistable bundles. 

$\chi ( E(r) ) = \sum _i a_i r^{n-i} $
$P_E (r) = \chi(E(r))/ a_0 $ and $\mu (E) = a_1/ a_0 $. 
Depending on the line bundle, then  are 2 different notions of stability,

$ E$ is (semi)-stable if and only if, for any coherent subsheaf $ F \rightarrow  E$, 
we have Geiseker-stability if $P_F ( r) \leq P_E (r) $ for all $r$ sufficiently large, 
and slope stable if $ \Gamma(F) \leq \Gamma(E) $. 

[[WHAT IS GAMMA? ]] 

In the case where $ X $ is a compact Riemann surface 
\begin{theorem}
    (Narasimhan- Seshadri) 
    An indecomposible holomorphic bundle $E$ is slope stable if and only if there is a unitary connection 
    on $E$ having constant central curvature $ F = -2 \pi i \mu (E) $ such a connection unique up to isomorphism. 
\end{theorem}
% section (end)

\bibliographystyle{plain}
\bibliography{}


\end{document} 

