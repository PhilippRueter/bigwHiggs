% 
\IfFileExists{/u/r/dw580/Qspace/mypreamble}{ \input{/u/r/dw580/Qspace/mypreamble} }{ \input{/home/waalge/Qspace/mypreamble} } % Work and Laptop versions

\title{Hitchin's equations: four dimensionsal motivation from Yang Mills theory, and reduction in two dimensions.}
\author{Raul Sanchez Galan}
\date{}

\begin{document} 
\maketitle

\section{Principal bundles} % (fold)
Let lie group $G$ act on manifold $P$.
Suppose $G$ acts freely on $P$. 
Let $M = P/G $. 
The projection $ \pi: P \rightarrow M $ is a submersion such that there exists a good cover $U_i $ of $M$ which trivialises $P$. 
That is $ \pi^{-1} ( U_i ) \cong U_i \times G $. 

Let $ \rho : G \rightarrow \mathrm{GL} ( V) $ be some representation. 
The associated vector bundle is defined to be $P \times V / \sim$. 
Where $ (p,v) \sim ( pg, \rho(g^{-1}) v ) $

Example: 
The adjoint representation 
$ \mathrm{Ad}_g : G \rightarrow  G $ by $ a \mapsto gag^{-1} $. 
Something about $ G \rightarrow  \mathrm{Aut}(G) $ 
the derivative $ \mathfrak{ad}_g : \mathfrak{g} \rightarrow  \mathfrak{g} $ 

A connection on $ P$ is a 1-form $\omega$ with values in $ \mathfrak{g} $ such that
$R_g$ is right multiplication by $ g$ 
$ R_g ^* \omega = \mathfrak{ad} _{g^{-1}} (\omega)$
$ \omega(\mathfrak{r} (X)) $ for vector field $X \in \mathfrak{G} $

Relating to the covariant derivative given by Charles, need to have (locally) a one form on $M$ with values in $ \mathfrak{g} $. 

Take a local trivialisation $ \{U_\alpha \} $, 
and set $ A_\alpha = s^* (\omega) $ 
where $ s: U_\alpha \rightarrow U_\alpha \times G $ is a section. 
The covariant derivative is locally defined by $ \nabla = d + A $.

The curvature of $ \nabla : \Omega^0 ( E) \rightarrow  \Omega^1 ( E) $
But induces map $d_A$ extending to the exterior algebra. 
So define curvature $F_A = d_A ^2 : \Omega^0 ( E ) \rightarrow \Omega^2(0)$ 

In local coordinates we can write 
$\nabla _i = \frac{\partial}{\partial x_i } + A_i $ where 
$ A_\alpha = \sum A_i dx_i $

Yang mills: 
\begin{equation}
    YM(A) = \int _{} | F_A | ^2 d \mu 
\end{equation}

The Euler Lagrange of the Yang Mills: 
\begin{equation}
     d_A F_A = 0 
\end{equation}
which is essentially the Bianchi identity 
\begin{equation}
    d_A * F_A = 0 
\end{equation}
and is 2nd order PDE on $A$.
Minimum of YM with finite action are called instantons.  

The Hodge star $*$ requires a metric to be defined. 
For now we restrict to $ M = \mathbb{R}  ^4 $ with the standard metric. 

\begin{equation}
    YM ( E) \geq 8 \pi ^2 c_2 (E) 
\end{equation}
where 
\begin{equation}
    c_2 (E) = \frac{1}{8 \pi^2} \int_M \mathrm{tr} (F_A ^2) 
\end{equation}
have 
\begin{align}
    * : \Omega^2(\mathfrak{ad}P) & \rightarrow  \Omega^2(\mathfrak{ad}P) \\
\end{align}
$ * ^2 = ( -1) ^{n(n-k) } $ 
as $ * ^2 = 1 \Rightarrow \Omega^2 () ... = \Omega^+ \oplus \Omega^- $. 
$ F_A = F_A ^+ + F_A ^- $ 
So in that case on minimum critical points we have that $ F_A = ^* F_A$ 
which is only first order in $A$. 
This is if and only if $A$ dual and if and only if $A$ is anti-self dual. 

Note: Some issue computing Chern classes over the real bundles?

Let $G$ be a compact group acting $ \mathbb{R} ^4$. 
$ F = F_{ij} dx_i \wedge dx_j $.
Then the self dual equations become:
\begin{align}
    F_{12} &= F_{34}  &
    F_{13} &= F_{42}  &
    F_{14} &= F_{23}  &
\end{align}

Suppose $ A $ is translation invariant in $x_3$ and $x_4$.
Relabel $ A_3 = \varphi_1 $ and $A_4 \varphi_2$.
Then on $ \mathbb{R} ^2 $ consider the connection $ A = A_1 dx_1 + A_2 dx_2 $. 

ALGEBRA

$\varphi = \varphi_1 + i \varphi_2$
$ F_{12} = i/2 [ \varphi , \varphi ^* ] $
$[ \nabla_1 + i \nabla_2 , \varphi ] = 0 $ 
$ z = x_1 + ix_2$ 
introduce 
$ \Phi = \frac{1}{2} \varphi dz \in \Omega^{1,0} ( \mathbb{R} ^2 , \mathfrak{ad}P \otimes \mathbb{C} )$
$\Phi^* = \frac{1}{2} \varphi * d\bar{z}$

the equations become 
\begin{align}
    F = - [ \Phi , \Phi^* ] \\
    \bar{\partial} _A \Phi = 0 
\end{align}

These are invariant under changes of coordinates and trivialisations of $ \Omega^1(\mathfrak{ad} P ) $

Observe that the second equation syas that $\Phi $ is a holomorphic section of $ \Omega^{1,0} ( \mathfrak{ad} P \otimes \mathbb{C}) $.

Examples: 
$ M $ a compact Riemann surface. 
$ G$ a compact Lie croup. 
then $ \Phi = 0 \rightarrow F_A = 0 $ unitary flat connections are in one to one correspondence with stable holomorphic bundles. 

$ g = h dz \wedge d \bar{z} $ compatible with the ex structure. 
$ G = \mathrm{U} ( 1) $. 
Take the Levi- Civita connection. 
Induces a connection on $K$ by $ d + A $, 
on $ K^{1/2} $ by $ d + 1/2 A $. 
On $ V = K^{1/2} \oplus K^{-1/2} $ have direct sum of two connections. 

Define $ \Phi = \begin{array}{rr}
    0 & 0 \\
    1 & 0 \\
\end{array} \in \Omega ^{1,0} ( \mathrm{End} (V) ) $ 

Have $ \mathrm{Hom} ( K^{1/2} , K^{-1/2} ) \cong K ^{-1} $ so $1$ denotes the canonical section of $ K^{-1} \otimes K $ 

$F_A = - 2h dz d \bar{z} $ and $ \mathrm{Ricci}_g = -2 g $ 
We are restricted to negative sectional curvature. 




% section (end)





\bibliographystyle{plain}
\bibliography{}


\end{document} 

