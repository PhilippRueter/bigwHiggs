% 
\documentclass[10pt]{article}

\usepackage[margin=1in]{geometry}  % set the margins to 1in on all sides
\usepackage{graphicx}              % to include figures
\usepackage{amsmath}               % great math stuff
\usepackage{amsfonts}              % for blackboard bold, etc
\usepackage{amsthm}                % better theorem environments
\usepackage{ulem}                  % underline emphasize
\usepackage{tikz}                  % Graphics
\usepackage{tabularx}                  % tables
\usetikzlibrary{matrix}
\usepackage{enumitem} 
\setlist[enumerate]{topsep=0pt,itemsep=-1ex,partopsep=1ex,parsep=1ex}
\usepackage[all]{xy}

\theoremstyle{plain}
\newtheorem{theorem}{Theorem}[section]
\newtheorem{lemma}[theorem]{Lemma}
\newtheorem{proposition}[theorem]{Proposition}
\newtheorem{corollary}[theorem]{Corollary}
\newtheorem{conjecture}[theorem]{Conjecture}
\newtheorem{definition}[theorem]{Definition}
\newtheorem{example}[theorem]{Example}
\newtheorem{remark}[theorem]{Remark}
\newtheorem{question}{Question}[section]

\newtheorem*{theorem*}{Theorem}
\newtheorem*{lemma*}{Lemma}
\newtheorem*{proposition*}{Proposition}
\newtheorem*{corollary*}{Corollary}
\newtheorem*{conjecture*}{Conjecture}
\newtheorem*{definition*}{Definition}
\newtheorem*{example*}{Example}
\newtheorem*{remark*}{Remark}
\newtheorem*{question*}{Question}



\title{Easy examples of Higgs Bundles and Hitchin Equations}
\author{}
\date{}

\begin{document} 
\maketitle
\Large
\section{The motivation} % (fold)

We wish to relate three distinct families of objects over a given manifold:
representations $ \pi_1 ( S) \rightarrow  \mathrm{Sl}(2, \mathbb{C} ) $;
harmonic bundles $ ( E, \nabla, h )$;
and Higgs bundles $(E, \varphi)$.

% section (end)


\section{Setting} % (fold)

Let $S$ connected closed oriented surface $ g \geq 2 $ and $X$ a Riemann surface over $ S$. 

\begin{definition}
    A Higgs bundle on $X$ is a pair $(E, \varphi)$ 
    where $E \rightarrow  X $ is a rank $2$ holomorphic vector bundle 
    with trivial determinant and $ \varphi \in H^0 ( X, K \otimes \mathrm{End} _0 (E) ) $.
\end{definition}

\begin{definition}
    A Higgs bundle $(E, \varphi)$ is stable if for any $L \subset E $ such that $ \varphi(L) \subset L \otimes K $ 
    we have $\mathrm{deg}(L) < 0 $. 
    Such $L$ are called $\varphi$ invariant.
    If $ E = \bigoplus_{i = 1} ^2  ( E_i , \varphi _i ) $ where the components are stable, then say $(E, \varphi)$is polystable.
\end{definition}

\begin{theorem}
 (Hitchin 87) 
    If $( E, \varphi) $ is polystable then there exists hermitian metric $H$ on $E$ such that the Chern connection $ A $ of $H$ satisfies 
    \begin{equation}
        F _A + [ \varphi , \varphi ^{*_H} ] = 0  
    \end{equation}
    (This is the Hitchin equation. ) 
\end{theorem}

[[ CHERN CONNECTION ]] [[ H HERMITIAN ON E OR M ?? Whats is *H ]]


Remark ... 
ALGEBRA 

Once you have a solution to the Hitchin Equation 
$\nabla = A +\varphi + \varphi^* $ connection on $E$ . 

$(A, \varphi)$ solution $\Leftrightarrow$ $\nabla $ is a flat connection. 

Get a representation $ \mathrm{hol}_\nabla : \pi(S) \rightarrow \mathrm{SL}(2, \mathbb{C} ) $. 

Moreover the fact that $ \varphi$ is holomorphic is equivalent to $H$ being a harmonic metric. 



% section (end)
 

\section{Example 1} % (fold)

$(E, \varphi)$ is a stable Higgs bundle iff $E$ is a stable vecotr bundles. 

Solving hitchins equations for $ (E, 0 ) $ means finding a metric $H$ such that 
\begin{align}
    F_A + [0,0^*] = 0 \\
    F_A = 0 
\end{align}
The Chern connection is already flat. 

We recover Narasimhan-seshadin
\begin{equation}
    \left\{ E \mbox{ rank 2, triv det, stable holom vb} \right\} \leftrightarrow 
    \left\{ \mathrm{Rep}( \pi_1 ( S) , \mathrm{SU}(2) ) \right\} 
\end{equation}

Relating the Moduli space $ \mathcal{M} _{0,2} (X)$ the Doulbeuat moduli. 
Complex structure with a complex compact submanifold. 
\begin{equation}
    M_{0,2} (X) \rightarrow T^* M_{0,2} (X) \xrightarrow{ \mbox{open, dense} } \mathcal{M} _{0,2}
\end{equation}


% section (end)

\section{Hyperbolic Geometry and Teichmuller theory} % (fold)

Recall that hyperbolic space $ \mathbb{H} = \{ (x,y) \in \mathbb{R}  |~ y> 0 \} $
has hyperbolic metric $ g = \frac{dx \wedge dy }{ y^2} $.
It has constant sectional curvature $-1$. 
The orientation preserving automorphism group is $ \mathrm{Aut}^+ ( \mathbb{H} ) = \mathrm{PSL}(2, \mathbb{R} ) $. 
How does one construct $ \mathbb{H}$-structures on $S $. 

[[Let $ \Gamma < \mathrm{PSL}(2, \mathbb{R}) $ ]]


We define the following three spaces associated to $S$.
The Fuchisan space $ \mathcal{D}F$ is 
\begin{align}
    \mathcal{D} F ( \pi_1 ( S) , \mathrm{PSL}_2 \mathbb{R} ) ) & = \\
    \{ \rho : \pi ( S)  \rightarrow  \mathrm{PSL}_2( \mathbb{R} ) & | \mbox{ injective, discrete} \} /\mathrm{PSL}_2 (\mathbb{R} ) 
\end{align}
The Fricke space $\mathcal{F}(S)$ is
\begin{align}
    \mathcal{F} ( S)  = \left\{ ( Y, h, f) |~ h \mbox{ hyperbolic metric } f:S \rightarrow Y \right\} / \mathrm{Diff}_0 ( S)  
\end{align}
The Teichmuller space $\mathcal{T}(S)$ is
\begin{align}
    \mathcal{T}(S)  = \left\{ ( X, l ) |~ l: S \rightarrow X  \right \}  / \mathrm{Diff}_0 ( S)  
\end{align}

There are maps relating these three spaces ad the spaces. 
cf Killing Hopf, Uniformization (Poincare Koebe) 

\begin{theorem}
As smooth manifolds $ \mathcal{D} F (S) = \mathcal{F} (S) = \mathcal{T} (S) = \mathbb{R}  ^{ 6g-6} $
\end{theorem}

\begin{theorem}
 (Wolf) 
    Fix $ X \in \mathcal{T} (S) $. 
    Then get homeomorphism 
    \begin{equation}
        \mathcal{F} (S) \rightarrow H^0 (X, K^2) 
    \end{equation}
\end{theorem}

Fix $ X \in \mathcal{T} ( S) $ choose a sqare root of $ K$ 
Take short exact sequence
\begin{equation}
    0 \rightarrow  \mathbb{Z} _2 \rightarrow \mathcal{O}^* \rightarrow \mathcal{O}^* 
\end{equation}

Let $ E = K^{1/2} \oplus K^{-1/2}$ which is holomorphic of rank 2. 
Note that this is unstable.
$ \mathrm{deg} ( K^{1/2} ) = g -1  > 0 $. 

Choose 
\begin{equation}
    \varphi = \begin{array}{rr}
        0 & 0 \\
        1 & 0 \\
    \end{array} \in H^0 ( X, K \otimes \mathrm{End}_ 0 ( E))
\end{equation}
The only $ \varphi$-invariant subbundle is $ 0 \oplus K^{-1/2} $ 
which has degree $ 1-g < 0 $. 
Thus $ ( E, \varphi ) $ is a stable higgs bundle. 

Can now apply Hitchins theorem and get $ H$ hermitian metric on $E$, 
such that the chern connection satisfies $F_A + [\varphi, \varphi^* ] = 0 $ 

Remark: 
E is decomposible. 

\begin{align}
    \mathcal{M} _ {0,2} ( X) & \rightarrow H^0 (X, K ^2) \\
    ( K ^{1/2} \oplus K^{-1/2} , \left(\begin{array}{rr}
        0 & \alpha \\
        1 & 0 \\ 
    \end{array} \right) \mapsto \alpha 
\end{align}


% section (end)




\bibliographystyle{plain}
\bibliography{}


\end{document} 

