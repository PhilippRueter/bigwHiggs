\documentclass[10pt]{article}

\usepackage[margin=1in]{geometry}  % set the margins to 1in on all sides
\usepackage{graphicx}              % to include figures
\usepackage{amsmath}               % great math stuff
\usepackage{amsfonts}              % for blackboard bold, etc
\usepackage{amsthm}                % better theorem environments
\usepackage{ulem}                  % underline emphasize
\usepackage{tikz}                  % Graphics
\usepackage{tabularx}                  % tables
\usetikzlibrary{matrix}
\usepackage{enumitem} 
\setlist[enumerate]{topsep=0pt,itemsep=-1ex,partopsep=1ex,parsep=1ex}
\usepackage[all]{xy}

\theoremstyle{plain}
\newtheorem{theorem}{Theorem}[section]
\newtheorem{lemma}[theorem]{Lemma}
\newtheorem{proposition}[theorem]{Proposition}
\newtheorem{corollary}[theorem]{Corollary}
\newtheorem{conjecture}[theorem]{Conjecture}
\newtheorem{definition}[theorem]{Definition}
\newtheorem{example}[theorem]{Example}
\newtheorem{remark}[theorem]{Remark}
\newtheorem{question}{Question}[section]

\newtheorem*{theorem*}{Theorem}
\newtheorem*{lemma*}{Lemma}
\newtheorem*{proposition*}{Proposition}
\newtheorem*{corollary*}{Corollary}
\newtheorem*{conjecture*}{Conjecture}
\newtheorem*{definition*}{Definition}
\newtheorem*{example*}{Example}
\newtheorem*{remark*}{Remark}
\newtheorem*{question*}{Question}



\title{Stability of Higgs Bundles}
\author{Juultje Kok}
\date{}

\begin{document} 
\maketitle

\begin{remark}
	All the statements in the notes of this talk can be found in (paragraph 3) from Hitchins `87 paper `The self-duality equation on a Riemann surface' \cite{Hitchin}.
\end{remark}

\section{Higgs Bundles}
\begin{definition}
    Let $M$ be a compact Riemann surface. A $\mathrm{GL}_n $-Higgs bundle is pair $(V, \varphi)$ where $ V$ is a holomorphic rank $n$ vector bundle, and $ \varphi$ is a holomorphic section of $ \mathrm{End}(V) \otimes K $. Here we denote by $K$ the canonical bundle on $M$. 
\end{definition}

One can also define $G$-Higgs bundles for other groups $G \subseteq GL_n$ (to be more precise: real reductive Lie groups). In that case there are extra conditions on $V$ and $\varphi$ reflecting the properties of the group. For example, a $\mathrm{SL}(2,\mathbb{C})$-Higgs bundle consists of a vector bundle of rank 2 with trivial determinant and $\varphi \in H^0(\mathrm{End}_0(V)\otimes K)$, where $\mathrm{End}_0(V)$ are the trace-free endormorphisms of $V$. See also Remark \ref{G=SU(2)}.
\begin{remark}
Usually we view $\varphi \in H^0 ( \mathrm{End} (V) \otimes K)$ as a morphism $\varphi: V \rightarrow V \otimes K$.
\end{remark}

\begin{definition}
    Two Higgs bundles $(V, \varphi) $ and $(V', \varphi')$ are isomorphic if 
    there exists an isomorphism of vector bundles $ \psi: V \rightarrow  V'$ such that the following diagram commutes: 
    \[
    \begin{xymatrix}{
    V \ar[d]^{\phi}\ar[r]^{\psi}&V' \ar[d]^{\phi'}\\
    V\otimes K \ar[r]^{\psi\otimes\mathrm{id}}&V'\otimes K
	}\end{xymatrix}
    \]
\end{definition}

\subsubsection*{From a solution of Hitchins equations to a Higgs Bundle}
Recall that a solution of Hitchins equations is given by a pair $(A,\varphi)$ such that 
\begin{align}
	F_A + [\varphi,\varphi^*] = 0, \\
	\overline{\partial}_A\varphi = 0. \label{holomorphic}
\end{align}
Here $A$ is a connection on a principal $G$-bundle $P$, and $\varphi \in \Omega^1 ( M , \mathrm{ad} ( P) \otimes \mathbb{C} )$ where we denote by $\mathrm{ad} ( P) = (P \times \mathfrak{g})/\mathrm{ad}$ the vector bundle associated to the adjoint representation of $G$ on the Lie algebra $\mathfrak{g}$. The connection $A$ induces via $\overline{\partial}_A$ a holomorphic structure on the vector bundle $V = (P \otimes \mathbb{C} ^n) / G$ associated to $P$ via the standard representation of $G$. Note that $\Omega^1 ( M , \mathrm{ad} ( P))$ is defined as 
\[
\Omega^1 ( M , \mathrm{ad} ( P)\otimes \mathbb{C}) := H^0 ( M , ( \mathrm{ad} (P) \otimes \mathbb{C} ) \otimes K). 
\]
So if we show that $\mathrm{ad} ( P) \otimes \mathbb{C} \cong \mathrm{End}(V)$, then $(V,\varphi)$ is a Higgs bundle (equation (\ref{holomorphic}) then gives us that $\varphi$ is a holomorphic section of $\mathrm{End}(V)\otimes K$.

\begin{lemma}
	If $V$ is the associated vector bundle of $P$, then
	\[
	\mathrm{ad} ( P) \otimes \mathbb{C} \cong \mathrm{End}(V).
	\]
\end{lemma}
\begin{proof} Note that $(\mathrm{ad}P \otimes \mathbb{C} )_x = (P_x \times \mathrm{Mat}_n(\mathbb{C}))/G$ and $\mathrm{End}(V)_x = \mathrm{End}(V_x)$. Recall also that if you choose a point $p$ in a fiber $P_x$, then every point $q$ in the fiber $P_x$ can be expressed as $p\cdot g$ for a unique $g \in G$. Consider the following map on fibers: 
	\begin{align*}
	(\mathrm{ad}P \otimes \mathbb{C} )_x & \rightarrow  \mathrm{End}(V) _x \\
	[p, X ] & \mapsto ([pg, v] \mapsto [pg, g^{-1} X g v] ) 
	\end{align*}
It is easy to check that this map is well-defined and induces an isomorphism of vector bundles. 
\end{proof}

\begin{remark}\label{G=SU(2)}
	Note that we have the following isomorphisms of Lie algebras:
	\begin{equation}
	\mathfrak{so}(3) \otimes \mathbb{C} \cong \mathfrak{su}(2) \otimes \mathbb{C}  \cong \mathfrak{sl}(2, \mathbb{C} ) = \{ X \in \mathrm{Mat}(2, \mathbb{C} )| ~ \mathrm{tr}(X) = 0 \}.
	\end{equation}
	So a Higgs bundle coming from a $SO(3)$-solution of Hitchins equations, consist of a vector bundle of rank 2 and a trace-free Higgs field $\varphi \in H^0(\mathrm{End}_0(V)\otimes K)$.
\end{remark}

\section{Stability} % (fold)

\begin{definition}
    Let $( V, \varphi) $ be a Higgs bundle. Then a subbundle $ E \subset L$ is called $ \varphi$-invariant (or a Higgs subbundle) if $ \varphi(E) \subset E \otimes K$. 
\end{definition}
\begin{definition}
    A Higgs bundle $( V, \varphi) $ is called stable if $\mu(E) < \mu(V) $ for all $\varphi$-invariant subbundles $E \subseteq V$. 
    Recall that the slope of a bundle is defined as $ \mu(E) : = \frac{\mathrm{deg} ( \Lambda^nE)}{\mathrm{rank}(E)} $.
\end{definition}

One of the reasons why we are interested in stable Higgs bundles, is the following theorem: 
\begin{theorem}
	There is a 1-1 correspondence between $\mathrm{SO}(3)$-solutions of Hitchins equations up to gauge equivalence and the moduli space of stable ($\mathrm{SO}(3)$-)Higgs bundles. 
\end{theorem}

\begin{proof}
	See Theorem 2.1 and Theorem 4.3 in \cite{Hitchin}.
\end{proof}

\begin{example}\label{ex:Vstable}
	A Higgs bundle $(V,0)$ is stable if and only if $ V$ is stable as a vector bundle. 
\end{example}
Namely, every subbundle is invariant under the zero Higgs field, and hence in this case the notion of stability as a Higgs bundle is equivalent to the notion of stability as a vector bundle. In fact, if $ V$ is a stable vector bundle, then any $\varphi$ gives rise to a stable Higgs bundle $(V,\varphi)$. 
Furthermore, using the fact that $\mathrm{End}(V)=\mathbb{C}$ for every stable vector bundle $V$, one can easily show that if $V$ is a stable vector bundle, then $(V, \varphi)$ and $(V, \varphi') $ are isomorphic as Higgs bundles if and only if $ \varphi = \varphi'$. 

This last observation can be used to see that the cotangent bundle of the moduli space of stable vector bundles $T^* \mathcal{M}_{\mathrm{Vect}}$ sits inside the moduli space of stable Higgs bundles. 
Namely, using Serre-Duality one can show that the tangent space at a point $[V]$ of $\mathcal{M}_{\mathrm{Vect}}$ is given by:
\begin{align*}
	T_{[V]}\mathcal{M}_{\mathrm{Vect}} &\cong \mathrm{Ext}(V,V)\\
	&\cong H^1(\mathrm{End}(V))\\
	&\cong H^0(\mathrm{End}(V)\otimes K)^*.
\end{align*}	
Since the isomorphism class of Higgs bundles $(V,\varphi)$ with $V$ a stable vector bundle is completely determined by the Higgs field, we get a bijection:
	\begin{align*}
	\mathcal{M}_{\mathrm{Higgs}} \supset \{ ( V,\varphi) | \mbox{ $V$ stable} \} &\longleftrightarrow T^* \mathcal{M}_{\mathrm{Vect}} \\
	( V, \varphi) &\longmapsto \varphi. 
	\end{align*}
	
\begin{remark}
	Similar to the case of stable vector bundles, one can also show that for a stable Higgs bundle every endormorphism is a scalar multiplication. See Proposition 3.15 in \cite{Hitchin}.
\end{remark}
	
\begin{example}
	Let $M$ be a Riemann surface with $g>1$. Then the Higgs bundle $(V,\varphi)$ given by $V = K^{1/2}\otimes K^ {-1/2}$ and $\varphi = \begin{pmatrix}0&0\\1&0\end{pmatrix}$ is a stable Higgs bundle. 
\end{example}
	
	Note that in this case 
	\[
	\varphi: K^{1/2} \otimes K^{-1/2} \longrightarrow K^{3/2}\otimes K^{1/2},
	\]
	so $1$ is just the identity morphism form $K^{1/2}$ to $K^{1/2}$. Since $\begin{pmatrix}0&0\\1&0\end{pmatrix}\begin{pmatrix}v_1\\v_2\end{pmatrix}=\begin{pmatrix}0\\v_1\end{pmatrix}$, the only $\varphi$-invariant subbundle of $V$ is the line bundle $K^{-1/2}$. So we only need to check the stability condition for $K^{-1/2}$. Note that we have
	\[
	\deg(K^{-1/2}) = -\frac 12(2g-2) = -(g-1)<0.
	\]
	On the other hand we also have 
	\begin{align*}
		\Lambda^2(K^{1/2}\otimes K^{-1/2}) &\cong \Lambda K^{1/2} \otimes \Lambda K^{-1/2} \\
		&\cong K^{1/2}\otimes K^{-1/2}\\
		&\cong \mathcal{O}.
	\end{align*}
	Hence we can conclude that $\deg(K^{-1/2})<0 = \frac 12 \deg(\Lambda^2V)$, from which it follows that $(V,\varphi)$ is a stable Higgs bundle. Note that $V$ is \textit{not} a stable vector bundle, since $\deg(K^{1/2})>0$.
	\\
	 
To further illustrate the notion of stability, we will show the following result: 

\begin{lemma}\label{nostableonP1}
	There are no stable Higgs bundles of rank 2 on $\mathbb{P}^1$. 
\end{lemma}

For the proof of this statement we will need the following lemma: 
\begin{lemma}\label{lem:Hom(L1,L2)}
	If $L_1$ and $L_2$ are both line bundles on $X$, then we have: 
	\[
	\mathrm{Hom} ( L_1 , L_2 ) \cong H^0 ( L_1 ^* \otimes L_2 ).
	\]
\end{lemma}

\begin{proof}
	Take a morphism $\theta \in \mathrm{Hom}(L_1,L_2)$ and consider the following morphism: 
	\[
	\mathrm{id}_{L_1^*} \otimes \theta  : \mathcal{O}_X \cong L_1^* \otimes L_1 \longrightarrow L_1^* \otimes L_2.
	\]
	Since a linebundle is a locally free $\mathcal{O}_X$-module of rank 1, a morphism from $\mathcal{O}_X$ to another linebundle is completely determined by the image of the generator $1 \in \mathcal{O}_X(X)$. Hence $\mathrm{id}_{L_1^*} \otimes \theta$ determines a unique element $a_{\theta} \in H^0(L_1^*\otimes L_2)$.
\end{proof}

\begin{proof}[Proof of Lemma \ref{nostableonP1}]
	Let $V$ be a rank 2 vector bundle on $\mathbb{P} ^1 $. Since every vectorbundle on $\mathbb{P}^1$ splits as a direct sum of line bundles $\mathcal{O}(a_i)$, we have 
	\[
	V \cong \mathcal{O} (m) \oplus \mathcal{O} (n)
	\]
	for some intergers $m,n \in \mathbb{Z}$. Recall that the canoncial line bundle on $\mathbb{P}^1$ is given by $K\cong \mathcal{O}(-2)$. So we can represent a morphism $\varphi \in H^0(\mathrm{End}(V)\otimes K)$ as follows: 
	\[
	\varphi = \begin{pmatrix}
	a&b\\c&d
	\end{pmatrix}:\mathcal{O} ( m) \oplus \mathcal{O} ( n) \rightarrow  \mathcal{O} (m -2) \oplus \mathcal{O} (n-2).
	\]
	Without loss of generality we may assume that $n-m<0$. Using the lemma above, together with the fact that line bundles with negative degree have no global sections, this gives us the following:
	\begin{align*}
		a &\in H^0(\mathcal{O}(-2)) = 0 \\
		b &\in H^0(\mathcal{O}(m-n-2)) \\
		c &\in H^0(\mathcal{O}(n-m-2))=0 \\ 
		d &\in H^0(\mathcal{O}(-2))=0.
	\end{align*}
 	Hence $\varphi = \begin{pmatrix} 0&b\\0&0 \end{pmatrix}$, and $\mathcal{O}(m)$ is the only $\varphi$-invariant subbundle of $V$. Now note that we have 
 	\[
 	\Lambda^2(\mathcal{O} ( m) \oplus \mathcal{O} (n)) \cong \mathcal{O}(m) \otimes \mathcal{O}(n) \cong \mathcal{O} (m+n).
 	\]
 	Since $n-m<0$, this gives us 
 	\[
 	\mathrm{deg}(\mathcal{O}(m)) = m \geq \frac12(m+n) = \frac12\mathrm{deg}(\Lambda^2V).
 	\]
 	From this we can conclude that every Higgs bundles of rank 2 on $\mathbb{P} ^1$ is unstable.
\end{proof}
 

% section (end)

\section{Moduli} % (fold)


\begin{lemma}\label{lem:uniqueL}
If $V$ is unstable vector bundle of rank 2, 
then there exists a unique $L \subset V$ such that
$ \mathrm{deg}(L) \geq \frac{1}{2} \mathrm{deg} (\Lambda^2 V ) $ and $V$ is given by an extension of the form
\begin{equation}
    0 \rightarrow  L \rightarrow V \rightarrow  L^* \otimes \Lambda^2V \rightarrow 0 
\end{equation}
\end{lemma}
\begin{proof}
	In fact one can prove that for every unstable vector bundle of rank $n$, there exists a unique maximal destabilizing subbundle of largest slope. This is used to proof the uniqueness of the Harder-Narasimhan filtration. See for example Proposition V.1.13 in \cite{Kob}. For the last statement Hitchin refers to this paper \cite{AB} of Atiyah-Bott.
\end{proof}

\begin{proposition}\label{VinstableHiggs}
    Let $ g>1 $. A rank $2$ vector bundle $V$ occurs in a stable $\mathrm{SO}(3)$-Higgs bundle $(V, \varphi) $ if and only if one of the following holds: 
    \begin{enumerate}[label=(\roman*)]
        \item $V$ is stable,
        \item $V$ is semistable and $g>2$, 
        \item $V$ is semistable and $g=2$, and $V = U \otimes L$, with $U$ decomposable or $0 \rightarrow  \mathcal{O} \rightarrow U \rightarrow \mathcal{O} \rightarrow 0$, 
        \item $V$ is not semistable and $h^0 (L^{-2} \otimes \Lambda^2V \otimes K ) > 1 $,
        \item $V$ is not semistable and $h^0 (L^{-2} \otimes \Lambda^2V \otimes K ) = 1 $ and $V = L\oplus (L^* \otimes \Lambda^2V ) $
    \end{enumerate}
	Here $L\subseteq V$ is the unique line bundle with $\mathrm{deg}(L) \geq \frac{1}{2} \mathrm{deg} (\Lambda^2V ) $.
\end{proposition}

\begin{remark}
	In Lemma \ref{nostableonP1} we showed that there are no stable Higgs bundles in the case when $g=0$. Hitchin shows in his paper (Remark 3.2.iv) that there is only one possibility for a stable $\mathrm{SO}(3)$-Higgs bundle when $g=1$. Namely $(V,0)$ where $V$ is an extension $0\rightarrow \mathcal{O} \rightarrow V \rightarrow \mathcal{O}(-1) \rightarrow 0$. In that case $V$ is already a stable vector bundle, and since $\varphi$ must be trace-free, we have $\varphi=0$. 
\end{remark}

\begin{proof}[Idea of the proof of Proposition \ref{VinstableHiggs} ]
	It is clear that any stable vector bundle $V$ occurs in a stable pair (see Example \ref{ex:Vstable}). So assume that $V$ is not a stable vector bundle and let $L\subseteq V$ be the (unique) linebundle which violates the stability condition (see Lemma \ref{lem:uniqueL}). Consider the following exact sequence:
	\[
	 0 \rightarrow  F \rightarrow  \mathrm{End} _0 ( V) \otimes K \rightarrow L^{-2} \otimes \Lambda^2V \otimes K \rightarrow  0 
	\]
	Here $F$ is defined as the kernel of the map $\mathrm{End} _0 ( V )\otimes K \rightarrow L^{-2} \otimes \Lambda^2V \otimes K$. Using Lemma \ref{lem:Hom(L1,L2)} and Lemma \ref{lem:uniqueL} we have: 
	\begin{align*}
		H^0(L^{-2}\otimes \Lambda^2V\otimes K) &\cong \mathrm{Hom}(L,L^* \otimes \Lambda^2V\otimes K) \\
		&\cong\mathrm{Hom}(L,V/L \otimes K).
	\end{align*}
	So the map $H^0(\mathrm{End} _0 ( V) \otimes K)) \rightarrow H^0(L^{-2} \otimes \Lambda^2V \otimes K)$ is given by $\varphi \mapsto \pi \circ \varphi|_{L}$. Therefore we have:
	\begin{align*}
	H^0(F) &= \{ \varphi \in H^0(\mathrm{End}_0(V)\otimes K) \colon \pi \circ \varphi|_{L} = 0 \}\\
	&= \{ \varphi \in H^0(\mathrm{End}_0(V)\otimes K) \colon \varphi(L) \subseteq L \}.
	\end{align*}
	In other words, $H^0(F)$ consists of all the Higgs-field $\varphi$ for which $L$ is $\varphi$-invariant. Now consider the induced long exact sequence in cohomology: 
	\[
	 0 \rightarrow  H^0(F) \overset{f}{\rightarrow} H^0(\mathrm{End} _0 ( V) \otimes K) \overset{g}{\rightarrow} H^0(L^{-2} \otimes \Lambda^2V \otimes K ) \overset{\delta}{\rightarrow} H^1(F) \rightarrow \ldots 
	\]
	Using the exactness of the above sequence, we can conclude the following:
	\begin{align*}
		0 = \mathrm{ker}(\delta)= \mathrm{im}(g)
		&\iff H^0(\mathrm{End}_0(V)\otimes K) = \mathrm{ker}(g) = \mathrm{im}(f)\\
		&\iff H^0(\mathrm{End}_0(V)\otimes K) \cong H^0(F)\\
		&\iff \varphi(L)\subseteq L \text{ for all } \varphi \in H^0(\mathrm{End}_0(V)\otimes K).
	\end{align*}
	Now remember that $\mathrm{deg}(L) \geq \frac{1}{2} \mathrm{deg} (\Lambda^2V ) $. So if $L$ is $\varphi$-invariant for every $\varphi \in H^0(\mathrm{End}(V)\otimes K)$, then every pair $(V,\varphi)$ is an unstable Higgs bundle. On the other hand, if $V$ does not occur in a stable pair, then $(V,\varphi)$ is an unstable Higgs bundle for every $\varphi \in H^0(\mathrm{End}(V)\otimes K)$. Since $L$ is the unique subbundle of $V$ which violates the stability condition, this means that $L$ is $\varphi$-invariant for all $\varphi \in H^0(\mathrm{End}(V)\otimes K)$. Hence we can conclude that $\delta$ is injective if and only if $V$ does not occur in a stable pair. \\
	
	In other words, $V$ does occur in a stable pair if and only if the coboundary map $\delta$ is \textit{not} injective. The rest of the proof now boils down to analyzing the injectivity of $\delta$ in all possible cases. For example, if $V$ is not semistable, one can show that $H^1(F) \cong \mathbb{C}$, which immediately gives case $(iv)$. See Proposition 3.3 in \cite{Hitchin} for the remaining work. 
\end{proof}

% section (end)


\bibliographystyle{plain}
\bibliography{references}


\end{document} 
