% 
\documentclass[10pt]{article}

\usepackage[margin=1in]{geometry}  % set the margins to 1in on all sides
\usepackage{graphicx}              % to include figures
\usepackage{amsmath}               % great math stuff
\usepackage{amsfonts}              % for blackboard bold, etc
\usepackage{amsthm}                % better theorem environments
\usepackage{ulem}                  % underline emphasize
\usepackage{tikz}                  % Graphics
\usepackage{tabularx}                  % tables
\usetikzlibrary{matrix}
\usepackage{enumitem} 
\setlist[enumerate]{topsep=0pt,itemsep=-1ex,partopsep=1ex,parsep=1ex}
\usepackage[all]{xy}

\theoremstyle{plain}
\newtheorem{theorem}{Theorem}[section]
\newtheorem{lemma}[theorem]{Lemma}
\newtheorem{proposition}[theorem]{Proposition}
\newtheorem{corollary}[theorem]{Corollary}
\newtheorem{conjecture}[theorem]{Conjecture}
\newtheorem{definition}[theorem]{Definition}
\newtheorem{example}[theorem]{Example}
\newtheorem{remark}[theorem]{Remark}
\newtheorem{question}{Question}[section]

\newtheorem*{theorem*}{Theorem}
\newtheorem*{lemma*}{Lemma}
\newtheorem*{proposition*}{Proposition}
\newtheorem*{corollary*}{Corollary}
\newtheorem*{conjecture*}{Conjecture}
\newtheorem*{definition*}{Definition}
\newtheorem*{example*}{Example}
\newtheorem*{remark*}{Remark}
\newtheorem*{question*}{Question}



\title{Hitchin to Higgs and back}
\author{Fabian}
\date{}

\begin{document} 
\maketitle
\Large

\section{Hitchins to Higgs} % (fold)

Let $M$ be compact Riemann surface of genus $g > 1$ and group $ G  = \mathrm{U}(2) $.
Let $ V \rightarrow M $ be a unitary vector bundle of rank 2, that is, $V$ has a Hermitian metric, and admits action by $\mathrm{U}(2)$.
Let $\{U_{\alpha} \} $ be a trivialisation of $V$ with transition functions 
$ u_{\alpha \beta} : U_{\alpha \beta} \rightarrow  \mathrm{U}(2) $. 

Suppose $ (A, \varphi) $ be the solution to the Hitchin equation \ref{hitchineq} for $A \in \Omega^1(V)$.
In local coordinates we can express the connection $A$ as 
\begin{equation}
    \nabla _{ A} s = ds + A_{\alpha} s \\
\end{equation}
The transformation conditions dictate $ A_j = u_{\alpha \beta} ^{-1} A_{\alpha} u_{\alpha \beta}  +  u_{\alpha \beta} ^{-1} d u_{\alpha \beta}$. 
Over the cover then $ F_A |_{U_{\alpha} }  = d A_{\alpha} + 1/2 [A_{\alpha} , A_{\alpha}] $
$ \Phi \in \Omega ^0 (M , \mathrm{End}_0 V) $.


Hitchin's equations are
\begin{align}
    F_A + [ \Phi , \Phi^* ] &=0 \\
    d_A ^{0,1} \Phi &= 0 
\end{align}

\begin{theorem}
    Let $( A,  \Phi ) $ be a solution to Hitchens equations. 
    Let $ L \subset V $ that is $ \Phi $ invariant subbundle of rank 1. 
    Then $ \mathrm{deg} (L ) \leq \frac{1}{2} \mathrm{deg} ( \mathrm{det} ( V) ) $ 
\end{theorem}

Idea of proof. 

Let $\omega $ be a $2$form such that 
$ \int 1/2 \pi \omega  = 1$ 

$ s \in \Omega( M, L^* \otimes V ) = \Omega ^0 ( M , \mathrm{Hom}(L, V) ) $ 
Where $s$ is holomorphic since $L$ is holomorphic. 

Construct a connection $B$ on $L^* \otimes V$ such that 
$F(B) s = F(A) s - \mathrm{deg} ( L) \leq \omega + 1/2 \mathrm{deg}( \mathrm{det} V ) \leq \omega $ 
$ - [\Phi , \Phi ^* ] s + ... $ 
and 
$ \left< F(B) s, s \right> _{L^2} \geq 0 $ 

If $ \mathrm{deg}(L) > 1/2 \mathrm{deg} (\mathrm{det} V)$ then $ \int \left< F(B) s, s \right> < 0 $  so contradiction.

\begin{lemma}
 $ L \rightarrow M $ holomorphic line bundle. 
    Then there exists connection $ \vartheta$ on $ L $ such that $ F(\vartheta) = \deg ( L) \omega $ 
\end{lemma}

\begin{proof} 
    Let $ \vartheta $ be a any connection on $ L $. 
    Then 
    \begin{align}
        \int _M \frac{1}{2 \pi}   F_\vartheta = \deg L = \int \frac{1}{2 \pi } \deg L \cdot \omega 
    \end{align}
    ( Upto constant. )
    This implies $ [ \frac{1}{2 \pi} F_\vartheta] = [ ( \deg L \cdot \omega ]  $ 

    Get a second connection $ \vartheta ' =  \vartheta + \partial \rho $ for some $ \rho  \in C^\infty (M) $. 
    Then $ F(\vartheta' ) = F(\vartheta) - \partial \bar{ \partial} \rho $. 

    Then $ F(\vartheta ' ) = \deg L \cdot \omega $ if and only if 
    $ \partial \bar{\partial } =  F(\vartheta) - \deg L \cdot \omega $. 

    The latter is precisely the $ d \bar{d} $ lemma . 
    And since we have a Kahler manifold we can solve for $\rho$. 
\end{proof}

We have exact sequence of groups 
\begin{equation}
    0 \rightarrow\mathrm{U}(1)  \rightarrow U ( 2) \rightarrow  \mathrm{SO}(3) \rightarrow  -0 
\end{equation}

$SO(3) $ connection $A$ induces $ \hat{A} $ on $V$ such that 
\begin{equation}
    F (\hat{A} ) = F ( A) + 1/2 F( A_0 ) \mathrm{id} 
\end{equation}

$\deg \det V \omega $ 
connection on $L$ 
Choose $A_0 $ such that curvature is $ F(A_0 )  = \deg L \cdot \omega $ 

$ \left< d_B ^{0,1} s, s \right> $, 
The first term lies in $ \Omega^{1,0} ( L^* \otimes V ) $ while the latter term lies in $ \Omega^{0} ( L^* \otimes V ) $.

\begin{align}
   d^{0,1}  \left< d_B ^{0,1} s, s \right> = \left<  d_B ^{0,1} d_B ^{0,1} s , s \right> - \left< d_B ^{0,1} s ,  d_B ^{0,1} s \right> 
\end{align}
Have that $ F(B) = d^2 _B =  d_B ^{0,1} d_B ^{1,0} - d_B ^{1,0} d_B ^{0,1} $
As $s$ is holomorphic, ie $ d_B ^{0,1} s = 0 $

\begin{equation}
    \int_M \left< F(B) s , s \right>  \geq 0 
\end{equation}



% section (end)

\section{Stable Higgs to Hitchin } % (fold)

Let $ \mathcal{A} $ be the space of all $C ^{\infty}$ connections on stable Higgs bundle $(V, \Phi)$. 
$\mathcal{A} $ is an affine space that is modelled on $ \Omega ^1(\mathrm{End}(V) )$. 
Thus $ T _ A \mathcal{A} = \Omega^1 ( \mathrm{End} ( V) ) $. 
This has a natural symplectic structure 
\begin{equation}
    \omega ( Y_1, Y_2 ) | _A : = \int _M \mathrm{Tr} (Y_1 ( A) \wedge Y_2 ( A) ) 
\end{equation}
where $Y_i$ vector fields on $ \mathcal{A} $. 
Thus we have infinite dimensional symplectic manifold. 

Recall that for symplectic manifold $(X, \omega)$, 
we have associated $G$ acting on $X$, where $ L_x \omega = 0 $. 
For $ g \in \mathfrak{g} $ have the associated $z_g \in \Gamma ( TX ) $.
$\omega( \cdot , z_g ) = d H_g $ the action of the Hamiltonian if and only if 
there exists $ H_g$ for $ \forall g \in \mathfrak{g} $ 

Define the moment 
\begin{equation}
    \mu : M \rightarrow \mathfrak{g} ^ * 
\end{equation}
with $ \left< \mu (m) , g \right> = H_g ( m) $.

Let $ \mathcal{G} $ be the group of gauge transformations. 
That is a collection of functions $g_{\alpha} : U_{\alpha} \rightarrow  {G} $ defined on a trivialisations
satisfying transformation rule $ g_{\alpha} = u_{\alpha \beta} g_j u_{\beta \alpha }$. 
Such maps are called gauge transformations and are equivalently defined as sections $g \in \Gamma(\mathrm{Ad}) $.
 

The general idea is that Hitchin's equations are the moment map on the space of connections $\mathcal{A} \times \Omega^{1,0} ( \mathrm{ad}P \otimes \mathbb{C} ) $. 

For a connection $\nabla$, the gauge transformation $g$ induces the following transformation to $\nabla'$
\begin{equation}
    \nabla' = \nabla  - ( \nabla  g ) g^{-1} 
\end{equation}

Now see $ \varphi $ as an element of the Lie algebra of the Gauge group. 
$ \varphi \in \mathrm{Lie} ( \mathcal{G} ) = \Omega^0 (M, \mathrm{End} ( V)) $

For a connection $A$, have the associated vector field
$ z_\varphi ( A) = \frac{d}{dt} |_{t=0 } ( \nabla_A - \nabla{A} ( I + t \varphi ) ( I -t \varphi) ) = - \nabla_A \varphi $

The associated hamiltonian to $z_\varphi ( A)$ is  
$ f_\varphi (A) = - \int_M \mathrm{Tr} (F_A \wedge \varphi ) $

We see this by considering
$ d f_\varphi |_A (a) = - \frac{d}{dt} |_{t=0} \int_M \mathrm{Tr} ( F_{A+t a} \wedge \varphi) $
where $a \in T_A \mathcal{A} $ that is $ a \in \Omega^1(\mathrm{End}(V) ) $.  

We can write
$F_{A+a } = F_A + d_A a + \frac{1}{2} [a,a ]$. 
Commuting the differentiation and integration, 
and seeing that $ \frac{d}{dt}|_{t=0} F_{A+ta} = d_A a  $
we recover $ \omega ( a , z_\varphi) =  - \int_M \mathrm{Tr} (d_A a \wedge \varphi)  $ ie the Hamiltonian. 

The moment map
$\mu : \mathcal{A} \rightarrow  \mathrm{Lie}^* ( \mathcal{G} )$ 

Note that since there exists a natural pairing between $ \Omega ^0 ( \mathrm{End}(V))$ and $ \Omega ^2 ( \mathrm{End}(V)) $, so we can identify 
$ \mathrm{Lie}^*(\mathcal{G}) = \Omega ^2 ( \mathrm{End}(V))$ 


Now consider the action of $\mathrm{U} ( n) $ on $\mathrm{C} ^n $ 
$ \mathrm{End} (\mathbb{C} ^n) $ has hermitian product.
$h(A,B) = \mathrm{tr} ( AB^*) $,
$U(n) $ acts on $ \mathrm{End} ( \mathbb{C} ^n) $ by conjugation 

$\omega(A,B) = - \mathrm{IM}(\mathrm{tr}( AB^*)) $

$ \mu (A) = i/2 [ A, A^*] $ 

$ \Omega^{1,0} ( M, \mathrm{End}_0 V ) $

$\omega(\Phi, \Phi )  = .... $ 

$\mathcal{G} $ acts on $ \Omega^{1,0}( M, \mathrm{End} _0 (V) ) $

$ \mu(\Phi ) = [ \Phi , \Phi ^* ] $ 

$ \mathcal{G} $ acts on $ \mathcal{A} \times \Omega^{1,0} (M , \mathrm{End}_0 V ) $ 

$ \mu( A, \Phi) = F ( A') + [\Phi , \Phi ^* ] = 0 $ 

To solve Hitchin equation find $(A, \Phi) $ such that $ \mu( A, \Phi) = 0 $. 

Choose a Riemannain metric on $M$ so that we can consider $ \| \mu \| ^0 $ 

Then $ \| \mu \|_2 ^ 2 = \int _M \| F_A + [\Phi , \Phi^* ] \| ^2 $ 



% section (end)

[[ THIS IS QUITE A NOTATION HEAVY PIECE. A LOT OF ATTENTTION NEEDED ]] 



\bibliographystyle{plain}
\bibliography{}


\end{document} 

